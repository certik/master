\section{Finite Element Method}

Finite element method (FEM) is a general method for solving partial
differential equations. Below we explain how to apply it to the Schr\"odinger
equation.

\subsection{Weak Formulation of the Schr\"odinger Equation}

One particle Schrödinger equation is
\begin{equation*}
  \left(-{\hbar^2\over2m}\nabla^2 + V\right)\psi=E\psi\,.
\end{equation*}
We multiply both sides by a test function $v$
\begin{equation*}
  -\left({\hbar^2\over2m}\nabla^2\psi\right)v=(E-V)\psi v\,,
\end{equation*}
and integrate over the whole volume we are interested in
\begin{equation}
  \int-\left({\hbar^2\over2m}\nabla^2\psi\right)v\,\d V=\int(E-V)\psi v\,\d V\,,  \label{1}
\end{equation}
and using the vector identity
\begin{equation*}
  -\left(\nabla^2\psi\right)v=\nabla \psi\cdot \nabla v - \nabla\cdot\left((\nabla \psi)v\right),
\end{equation*}
we rewrite the left hand side of (\ref{1})
\begin{equation*}
  \int{\hbar^2\over2m}\nabla\psi\cdot\nabla v\,\d V=\int(E-V)\psi v\,\d V+\int{\hbar^2\over2m}\nabla\cdot\left((\nabla \psi)v\right)\,\d V\,,
\end{equation*}
now we apply Gauss Theorem
\begin{equation*}
  \int{\hbar^2\over2m}\nabla\psi\cdot\nabla v\,\d V=\int(E-V)\psi v\,\d V+\oint{\hbar^2\over2m}(\nabla \psi)v\cdot{\bf n}\,\d S\,,
\end{equation*}
and rewriting $\nabla\psi\cdot{\bf n}\equiv{\d\psi\over\d n}$
\begin{equation}
  \int{\hbar^2\over2m}\nabla\psi\cdot\nabla v\,\d V+ \int vV\psi\,\d V = \int E\psi v\,\d V + \oint{\hbar^2\over2m}{\d\psi\over\d n}v\,\d S\,,  \label{w}
\end{equation}
which is the weak formulation. The problem reads: find a function $\psi$ such that (\ref{w}) holds for every $v$.

\subsection{Finite Elements}

We choose a basis $\phi_i$ and substitute $\phi_i$ for $v$ and expand $\psi=\sum q_j\phi_j$
\begin{equation}
  \left(\int{\hbar^2\over2m}\nabla\phi_j\cdot\nabla\phi_i\,\d V+ \int\phi_iV\phi_j\,\d V\right)q_j = \left(\int E\phi_j\phi_i\,\d V\right)q_j +\oint{\hbar^2\over2m}{\d\psi\over\d n}\phi_i\,\d S\,,  \label{fem}
\end{equation}
which can be written in a matrix form
\begin{equation*}
  \left(K_{ij}+V_{ij}\right)q_j=EM_{ij}q_j+F_i\,,
\end{equation*}
where
\begin{eqnarray*}
V_{ij}&=&\int\phi_iV\phi_j\,\d V\,, \\
M_{ij}&=&\int\phi_i\phi_j\,\d V\,, \\
K_{ij}&=&{\hbar^2\over2m}\int\nabla\phi_i\cdot\nabla\phi_j\,\d V\,, \\
F_i&=&{\hbar^2\over2m}\oint{\d\psi\over\d n}\phi_i\,\d S\,. \\
\end{eqnarray*}
 Usually we set $F_i=0$.

We decompose the domain into elements and compute the integrals as the sum over elements. For example:
\begin{equation*}
  K_{ij}=\sum_{E\in elements} K_{ij}^E
\end{equation*}
where $K_{ij}^E$ is the integral over one element only
\begin{equation*}
  K_{ij}^{E}=\int{\hbar^2\over2m}\nabla\phi_j\cdot\nabla\phi_i\,\d V^{E}\approx \sum_{q=0}^{N_q-1}{\hbar^2\over2m}\,\nabla\phi_i(x_q)\cdot\nabla\phi_j(x_q)\, w_q|\det J(\hat x_q)|\,.
\end{equation*}
The integral is computed numerically using a Gauss integration: $x_q$ are Gauss points (there are $N_q$ of them), $w_q$ is the weight of each point, and the Jacobian $|\det J(\hat x_q)|$ is there because we are actually computing the integral on the reference element instead in the real space.
The surface integrals are computed similarly.

Finite elements programs usually have an assembly phase, where they need to
assemble all the global matrices, in our case:
\begin{equation*}
  \left(K_{ij}+V_{ij}\right)q_j=EM_{ij}q_j+F_i\,,
\end{equation*}
and then a solve phase, in our case this amounts to solve a large sparse
generalized eigenvalue problem.
