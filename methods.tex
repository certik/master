\section{Finite Element Method}

The finite element method (FEM) is the most widely used technique for solving partial
differential equations (PDE). In this section we explain how it can be applied to the 
Schr\"odinger equation.

\subsection{Weak Formulation of the Schr\"odinger Equation}

One-particle Schr\"odinger equation has the form
\begin{equation}\label{schPDE}
  \left(-\frac{\hbar^2}{2m}\nabla^2 + V \right) \psi=E\psi\ \ \ \mbox{in}\ \Omega,
\end{equation}
where $\Omega$ is a Cartesian square (in 2D) or cube (in 3D) 
centered at the origin of the coordinate 
system. The domain is chosen sufficiently large so that the decay of $\psi$
allows us to prescribe zero Dirichlet conditions on the boundary $\Gamma$.
Here $V$ is a given function in $\Omega$ representing the potential well.

We multiply both sides of (\ref{schPDE}) by a test function $v$
that is zero on $\Gamma$ and integrate over $\Omega$,
\begin{equation}
  \int_{\Omega}-\left({\hbar^2\over2m}\nabla^2\psi\right)v\,\d V=\int_{\Omega}(E-V)\psi
v\,\d V\,.  \label{1}
\end{equation}
%the usual treatment in the finite element method is to apply the first Green's
%identity, but it is useful to do things explicitely to better understand what
%is going on, so we apply the following vector identity
%$$
%  -\left(\nabla^2\psi\right)v=\nabla \psi\cdot \nabla v -
%\nabla\cdot\left((\nabla \psi)v\right),
%$$
%we rewrite the left hand side of (\ref{1})
%$$
%  \int{\hbar^2\over2m}\nabla\psi\cdot\nabla v\,\d V=\int(E-V)\psi
%v\,\d V+\int{\hbar^2\over2m}\nabla\cdot\left((\nabla \psi)v\right)\,\d
%V\,,
%$$
%now we apply Gauss Theorem
Applying to the left-hand side 
the Green's first identity (generalized integration by parts formula), 
we obtain 
$$
  \int_{\Omega}{\hbar^2\over2m}\nabla\psi\cdot\nabla v\,\d V=\int_{\Omega}(E-V)\psi
v\,\d V+\oint_{\Gamma}{\hbar^2\over2m}(\nabla \psi)v\cdot{\bf n}\,\d S\,.
$$
%and rewriting $\nabla\psi\cdot{\bf n}\equiv{\d\psi\over\d n}$
%\begin{equation}
%  \int{\hbar^2\over2m}\nabla\psi\cdot\nabla v\,\d V+ \int vV\psi\,\d V
%= \int E\psi v\,\d V + \oint{\hbar^2\over2m}{\d\psi\over\d n}v\,\d
%S\,,  \label{w}
%\end{equation}
Since $v \equiv 0$ on $\Gamma$, the integral over $\Gamma$ vanishes and 
we obtain the following weak formulation of (\ref{schPDE}):
\def\S{S}
Find a function $\psi\in\S$ such that 
\begin{equation}\label{w}
  \int_{\Omega}{\hbar^2\over2m}\nabla\psi\cdot\nabla v\,\d V=\int_{\Omega}(E-V)\psi
v\,\d V\ \ \ \mbox{for every} \ v \in \S.
\end{equation}
Here, $\S$ is a subspace of the standard Sobolev space $H^1(\Omega)$ 
corresponding to zero Dirichlet boundary conditions on $\Gamma$, i.e.,
$$
\S = \{v \in H^1(\Omega); \; v = 0 \ \mbox{on} \ \Gamma   \}.
$$
Recall that the function space $H^1(\Omega)$ consists of all functions defined in $\Omega$
which are square-integrable and have square-integrable (weak) gradients.

\subsection{Finite Element Method}

The domain $\Omega$ is covered with a finite element mesh consisting of $M$ 
non-overlapping triangles (in 2D) or tetrahedra (in 3D) $K_1, K_2, \ldots, K_M$. 
Any two elements can only share a common vertex, a common edge (along with the corresponding pair of 
vertices), or a common face (along with the edges and vertices corresponding to that face).

The finite element approximation takes place in a finite-dimensional subspace $S_n \subset \S$ 
consisting of continuous functions defined in $\Omega$ which are linear within every element $K_m$,
$m = 1, 2, \ldots, M$.
Let $\phi_i$, $i = 1, 2, \ldots, n$ be a basis of $\S_n$. The approximate solution 
$\psi$ is sought as a linear combination of these basis functions with 
unknown coefficients,
$$
\psi=\sum q_j\phi_j.
$$
Substituting this expansion into (\ref{w}) and using the basis functions 
$\phi_i$, $i = 1, 2, \ldots, n$ in place of the test function $v$, we
obtain a discrete problem
\begin{equation}
  \left(\int_{\Omega}{\hbar^2\over2m}\nabla\phi_j\cdot\nabla\phi_i\,\d V+
\int_{\Omega}\phi_iV\phi_j\,\d V\right)q_j = \left(\int_{\Omega} E\phi_j\phi_i\,\d
V\right)q_j.
\label{fem}
\end{equation}
This can be written in a matrix form
$$
  \left(K_{ij}+V_{ij}\right)q_j=EM_{ij}q_j\,,
$$
where
\begin{eqnarray*}
V_{ij}&=&\int_{\Omega}\phi_iV\phi_j\,\d V\,, \\
M_{ij}&=&\int_{\Omega}\phi_i\phi_j\,\d V\,, \\
K_{ij}&=&{\hbar^2\over2m}\int_{\Omega}\nabla\phi_i\cdot\nabla\phi_j\,\d V\,.
%F_i&=&{\hbar^2\over2m}\oint{\d\psi\over\d n}\phi_i\,\d S\,. \\
\end{eqnarray*}
The integrals in (\ref{fem}) are evaluated via element contributions 
over $K_1, K_2, \ldots, K_M$. For example,
$$
  K_{ij}=\sum_{m=1}^M K_{ij}^{(m)},
$$
where 
$$
  K_{ij}^{(m)}=\int_{K_m}{\hbar^2\over2m}\nabla\phi_j\cdot\nabla\phi_i\,\d
V^{E}\approx \sum_{q=0}^{N_q-1}{\hbar^2\over2m}\,\nabla\phi_i(x_q)\cdot\nabla\phi_j(x_q)\,
w_q|\det J(\hat x_q)|\,.
$$
The integral is computed numerically using the Gauss integration
technique: $x_q$
are Gauss points (there are $N_q$ of them), $w_q$ is the weight of
each point, and the Jacobian $|\det J(\hat x_q)|$ is there because we
are actually computing the integral on the reference element instead
in the real space. It is worth mentioning that for both triangular 
elements in 2D and tetrahedra in 3D, the Jacobian is constant.
%The surface integrals are computed similarly.

%The decomposition into elements is done using a so called finite element mesh
%and the type of the mesh and the basis (e.g. linear, quadratic
%polynomials, higher order polynomials) establishes the function space $\S$ from
%the weak formulation (\ref{w}). In our case, we have a Dirichlet boundary
%condition that fixes the function value of all eigenvectors to 0 at the
%boundary.

Finite elements programs usually have an assembly phase, where they
construct all the matrices, in our case:
$$
  \left(K_{ij}+V_{ij}\right)q_j=EM_{ij}q_j\,,
$$
and then a solve phase. In our case this amounts to solving a large sparse
generalized eigenvalue problem.
