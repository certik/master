Our ultimate goal is to develop a new code for real-space electronic structure
calculations within density functional theory combining the advantages
of ab-initio pseudopotentials and finite element method for calculating
electronic states and related quantities of complex non-periodic systems.
This complex task that people started to tackle quite recently
\cite{pask1, pask2, ortiz1, ortiz2}
entails a series of new partial problems to be solved.

We describe our motivation in detail in Chapter 2.

The related theory and full ab-initio derivation of all the equations
in finite elements, density functional theory and pseudopotentials
are summarized in Chapter 3.
This chapter is split into parts describing the basic ingredients of our
recently developed method: the subsections 3.1. -- 3.4. deal respectively,
with Dirac and Schr\"odinger equations, with the density functional theory and
Kohn-Sham equations, with ab-initio pseudopotentials including their separable
form, and with the finite element method.


Chapter 4 summarizes the results obtained so far within the
present thesis, explaining how the pieces of our code works and showing
the examples of finite-element method applications for spherically symmetric
potentials in relativistic and nonrelativistic DFT and for 2D and 3D
Schr\"odinger equation for symmetric and non-symmetric potentials.

Finally the appendix contains some useful derivations and explanations like
delta functions, variations, functional derivatives, dirac notation and more.
