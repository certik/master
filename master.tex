\documentclass[12pt,notitlepage]{report}
%\pagestyle{headings}
\pagestyle{plain}

\frenchspacing % aktivuje použití některých českých typografických pravidel

\usepackage[utf8]{inputenc} % nastavuje použité kódování, uživatelé Windows zamění latin2 za cp1250
\usepackage{a4wide} % nastavuje standardní evropský formát stránek A4
%\usepackage{index} % nutno použít v případě tvorby rejstříku balíčkem makeindex
%\usepackage{fancybox} % umožňuje pokročilé rámečkování :-)
\usepackage{graphicx} % nezbytné pro standardní vkládání obrázků do dokumentu

\usepackage[left=4cm]{geometry} % nastavení dané velikosti okrajů
\usepackage{amsmath}
\usepackage{braket}
\usepackage{verbatim}
\usepackage{listings}
\usepackage{url}
\usepackage[numbers]{natbib}

\input macros.tex

\def\cite#1{\citep{#1}}


%\newindex{default}{idx}{ind}{Rejstřík} % zavádí rejstřík v případě použití balíku index

%\title{Název práce}   % tyto dvě položky jsou zde v podstatě formálně, ve skutečnosti nejsou nikde
%\author{Jméno autora} % dále v dokumentu použity

%\date{}

\begin{document}

%%% Následuje první, úvodní, strana bakalářské práce. Jednotlivé položky nahraďte dle vlastních
%%% údajů. Změnit podle konkrétní délky jednotlivých položek můžete i zalomení řádků.
\begin{titlepage}
\begin{center}
\ \\

\vspace{15mm}

\large
Charles University in Prague\\
Faculty of Mathematics and Physics\\

\vspace{5mm}

{\Large\bf MASTER THESIS}

\vspace{10mm}

\includegraphics[scale=0.3]{images/logo.pdf}

\vspace{15mm}

%\normalsize
{\Large Ondřej Čertík}\\ 
\vspace{5mm}
{\Large\bf Calculation of electron structure in the framework of DFT in real
space}\\
\vspace{5mm}
Institute of Theoretical Physics\\ % doplňte název katedry či ústavu
\end{center}
\vspace{15mm}

\large
\noindent Supervisor: RNDr. Jiří Vackář, CSc. Institute of Physics, \\
\hbox{$\quad\quad\quad\quad\quad$}   Academy of Sciences, Czech Republic
\vspace{1mm}\\
%
\noindent Field of study: theoretical physics
\bigskip
\bigskip
\begin{center}
2008
\end{center}

\end{titlepage}

\normalsize
\setcounter{page}{2}
\ \vspace{10mm}

\noindent I would like to thank my supervisor RNDr. Jiří Vackář, CSc. for
his time spent in countless discussions enlightening me a lot of issues in
quantum mechanics and helping with this work. I thank RNDr. Ondřej
Šipr, CSc. for explaining me a number of theoretical problems and general
advice. I am deeply indebted to RNDr. Antonín Fejfar, CSc. and Ing. Jiří
Plešek, CSc. for moral support during my university years and in the
course of writing this diploma thesis. I also wish to thank Nissy Nevil
for
her proofreading and amiable encouragement. Finally, I thank Ing. Robert
Cimrman, Ph.D. and his sfepy (a finite element package).

\vspace{\fill}
\noindent I declare that I wrote the thesis by myself and listed all used
sources. I agree with making the thesis publicly available.

\bigskip
\noindent Prague, August 5, 2008 \hspace{\fill}Ondřej Čertík\\ % doplňte patřičné datum, jméno a příjmení

%%%   Výtisk pak na tomto míste nezapomeňte PODEPSAT!
%%%                                         *********


\newpage % přechod na novou stránku

%%% Následuje strana s abstrakty. Doplňte vlastní údaje.
\noindent
Title: Calculation of electron structure in the framework of DFT in real
space\\
Author: Ondřej Čertík\\
Department: Institute of Theoretical Physics\\
Supervisor: RNDr. Jiří Vackář, CSc.\\
Supervisor's e-mail address: vackar@fzu.cz\\

\noindent Abstract: In the present work we study ab-initio electronic structure
calculations in real space using density functional theory (DFT), finite
elements and pseudopotentials. We summarize the theory and full ab-initio
derivation of all equations in finite elements, density functional theory and
pseudopotentials, then we explain how our program works and we show results for
spherically symmetric potentials in relativistic and nonrelativistic DFT and
for 2D and 3D Schr\"odinger equation for symmetric and non-symmetric
potentials. \\

\noindent Keywords: Finite element method, Density functional theory

\vspace{10mm}

\noindent
Název práce: Výpočty elektronových struktur v rámci DFT v reálném prostoru\\
Autor: Ondřej Čertík\\
Katedra (ústav): Ústav teoretické fyziky\\
Vedoucí bakalářské práce: RNDr. Jiří Vackář, CSc.\\
e-mail vedoucího: vackar@fzu.cz\\

\noindent Abstrakt:  V předložené práci studujeme ab-initio výpočty
elektronových struktur v rámci teorie funkcionálu hustoty (DFT), konečných
prvků a pseudopotenciálů. Shrneme teorii a odvození všech potřebných vztahů v
metodě konečných prvků, teorii funkcionálu hustoty a pseudopotenciálů z prvních
principů, pak vysvětlíme jak náš program funguje a ukážeme výsledky pro
sféricky symetrický potenciál v relativistické i nerelativistické verzi DFT a
2D a 3D Schrödingerovu rovnici pro symetrické a nesymetrické potenciály. \\

\noindent Klíčová slova: metoda konečných prvků, teorie funkcionálu hustoty

\tableofcontents % vkládá automaticky generovaný obsah dokumentu

\newpage

\chapter{Introduction}

\input introduction.tex

\chapter{Motivation, Tasks}

\input motivation.tex

\chapter{Methods Used}

\input theory.tex
\input methods.tex

\chapter{Results}

\input results.tex

\chapter*{Appendix}
\addcontentsline{toc}{chapter}{Appendix}

%\input qft.tex
\input appendix.tex

%%% Seznam literatury
%%%
%%% Literatura se řadí abecedně. Úvádí se pouze literatura, na kterou se v textu odkazuje.
%%% Při odkazu na knihu se vždy uvádějí čísla stránek.

\bibliographystyle{plainnat}
\bibliography{master}


\end{document}
