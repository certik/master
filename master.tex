\documentclass[12pt,notitlepage]{report}
%\pagestyle{headings}
\pagestyle{plain}

\frenchspacing % aktivuje použití některých českých typografických pravidel

\usepackage[utf8]{inputenc} % nastavuje použité kódování, uživatelé Windows zamění latin2 za cp1250
\usepackage{a4wide} % nastavuje standardní evropský formát stránek A4
%\usepackage{index} % nutno použít v případě tvorby rejstříku balíčkem makeindex
%\usepackage{fancybox} % umožňuje pokročilé rámečkování :-)
\usepackage{graphicx} % nezbytné pro standardní vkládání obrázků do dokumentu

\usepackage[left=4cm]{geometry} % nastavení dané velikosti okrajů

%\newindex{default}{idx}{ind}{Rejstřík} % zavádí rejstřík v případě použití balíku index

\title{Název práce}   % tyto dvě položky jsou zde v podstatě formálně, ve skutečnosti nejsou nikde 
\author{Jméno autora} % dále v dokumentu použity

%\date{}

\begin{document}

%%% Následuje první, úvodní, strana bakalářské práce. Jednotlivé položky nahraďte dle vlastních
%%% údajů. Změnit podle konkrétní délky jednotlivých položek můžete i zalomení řádků.
\begin{titlepage}
\begin{center}
\ \\

\vspace{15mm}

\large
Charles University in Prague\\
Faculty of Mathematics and Physics\\

\vspace{5mm}

{\Large\bf MASTER THESIS}

\vspace{10mm}

\includegraphics[scale=0.3]{logo.eps}

\vspace{15mm}

%\normalsize
{\Large Ondřej Čertík}\\ 
\vspace{5mm}
{\Large\bf Calculation of electron structure in the framework of DFT in real
space}\\
\vspace{5mm}
Institute of Theoretical Physics\\ % doplňte název katedry či ústavu
\end{center}
\vspace{15mm}

\large
\noindent Supervisor: RNDr. Jiří Vackář, CSc. Institute of Physics, \\
\hbox{$\quad\quad\quad\quad\quad$}   Academy of Sciences, Czech Republic
\vspace{1mm}\\
%
\noindent Field of study: theoretical physics
\bigskip
\bigskip
\begin{center}
2008
\end{center}

\end{titlepage}

\normalsize
\setcounter{page}{2}
\ \vspace{10mm}

\noindent I would like to thank RNDr. Jiří Vackář, CSc., for many discussions,
enlightening a lot of issues in quantum mechanics to me and for his time
helping me with this work. I thank to RNDr. Ondřej Šipr, CSc. for explaining
me many theoretical problems and general advice. I very much thank Ing. Jiří
Plešek, CSc. for moral support during my university years and during writing
this thesis.

\vspace{\fill}
\noindent I declare that I wrote the thesis by myself and listed all used
sources. I agree with making the thesis publicly available.

\bigskip
\noindent Prague, August 5, 2008 \hspace{\fill}Ondřej Čertík\\ % doplňte patřičné datum, jméno a příjmení

%%%   Výtisk pak na tomto míste nezapomeňte PODEPSAT!
%%%                                         *********

\tableofcontents % vkládá automaticky generovaný obsah dokumentu

\newpage % přechod na novou stránku

%%% Následuje strana s abstrakty. Doplňte vlastní údaje.
\noindent
Název práce: Název bakalářské práce\\
Autor: Jméno autora\\
Katedra (ústav): Název katedry či ústavu\\
Vedoucí bakalářské práce: Jméno se všemi tituly, event. pracoviště\\
e-mail vedoucího: e-mailová adresa vedoucího\\

\noindent Abstrakt:  V předložené práci studujeme ... Uvede se abstrakt v rozsahu 80 až 200 slov. Lorem ipsum dolor sit amet, consectetuer adipiscing elit. Ut sit amet sem. Mauris nec turpis ac sem mollis pretium. Suspendisse neque massa, suscipit id, dictum in, porta at, quam. Nunc suscipit, pede vel elementum pretium, nisl urna sodales velit, sit amet auctor elit quam id tellus. Nullam sollicitudin. Donec hendrerit. Aliquam ac nibh. Vivamus mi. Sed felis. Proin pretium elit in neque. Pellentesque at turpis. Maecenas convallis. Vestibulum id lectus. Fusce dictum augue ut nibh. Etiam non urna nec mi mattis volutpat. Curabitur in tortor at magna nonummy gravida. Mauris turpis quam, volutpat quis, porttitor ut, condimentum sit amet, felis.\\

\noindent Klíčová slova: klíčová slova (3 až 5)

\vspace{10mm}

\noindent
Title: Název bakalářské práce v angličtině\\
Author: Jméno autora\\
Department: Název katedry či ústavu v angličtině\\
Supervisor: Jméno s tituly jako v české verzi, event. pracoviště\\
Supervisor's e-mail address: e-mailová adresa vedoucího\\

\noindent Abstract: In the present work we study ... Uvede se anglický abstrakt v rozsahu 80 až 200 slov. Lorem ipsum dolor sit amet, consectetuer adipiscing elit. Ut sit amet sem. Mauris nec turpis ac sem mollis pretium. Suspendisse neque massa, suscipit id, dictum in, porta at, quam. Nunc suscipit, pede vel elementum pretium, nisl urna sodales velit, sit amet auctor elit quam id tellus. Nullam sollicitudin. Donec hendrerit. Aliquam ac nibh. Vivamus mi. Sed felis. Proin pretium elit in neque. Pellentesque at turpis. Maecenas convallis. Vestibulum id lectus. Fusce dictum augue ut nibh. Etiam non urna nec mi mattis volutpat. Curabitur in tortor at magna nonummy gravida. Mauris turpis quam, volutpat quis, porttitor ut, condimentum sit amet, felis. \\

\noindent Keywords: klíčová slova (3 až 5) v angličtině

\newpage

%%% Následuje text bakalářské práce členěný do kapitol, které se číslují, označí názvy a graficky oddělí.
%%% Nedoporučuje se používat víc než dvě úrovně číslování kapitol, viz příklad níže.

\chapter{Název první kapitoly}

\section{Název první podkapitoly v první kapitole}

\section{Název druhé podkapitoly v první kapitole}

\chapter{Název druhé kapitoly}

\section{Název první podkapitoly v druhé kapitole}

\section{Název druhé podkapitoly v druhé kapitole}


%%% Seznam literatury
%%%
%%% Literatura se řadí abecedně. Úvádí se pouze literatura, na kterou se v textu odkazuje.
%%% Při odkazu na knihu se vždy uvádějí čísla stránek.

\begin{thebibliography}{99}
\addcontentsline{toc}{chapter}{Literatura}
 \bibitem{abraham-marsden}Abraham R., Marsden J. E.: {\em Foundations of Mechanics}, Addison-Wesley, Reading, 1985.
 \bibitem{derbes}Derbes D.: {\em Reinventing the wheel: Hodographic solutions to the Kepler problems}, Am. J. Phys. {\bf 69} (2001) 481--489.
 \bibitem{kvasnica}Kvasnica J.: {\em Teorie elektromagnetického pole}, Academia, Praha, 1985.
\end{thebibliography}

\end{document}
