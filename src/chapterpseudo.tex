\chapter{Pseudopotentials}

\section{Introduction}

Literature about pseudopotentials is unfortunately scattered among many
arcticles, so this section gives a review and should save the reader from a lot
of troubles.

\section{Hermitian Operators in Spherical Symmetry}

We show that every Hermitian operator $\hat V$ in the spherical symmetric
problem ($\hat V=R^{-1}\hat VR$) can be written in the form
$$\hat V=\sum_{lm}\ket{lm}\hat V_l\bra{lm}\no{lmexpansion}$$
where the operator $\hat V_l=\braket{lm|\hat V|lm}$ has matrix elements
$$\braket{\rho|\hat V_l|\rho'}=\bra{lm}\braket{\rho|\hat V|\rho'}\ket{lm}=
V_l(\rho,\rho')$$
{\bf Proof:} Matrix elements of a general Hermitian operator $\hat V$ are
$$\braket{{\bf r}|\hat V|\varphi}=
\int\braket{{\bf r}|\hat V|{\bf r'}}\braket{{\bf r'}|\varphi}\d^3r'=
\int V({\bf r},{\bf r'})\varphi({\bf r'})\d^3r'$$
where
$$V({\bf r}, {\bf r'})=\braket{{\bf r}|\hat V|{\bf r'}}$$ 
In spherical symmetry, we have 
$$\braket{{\bf r}|\hat V|\varphi}
=\braket{{\bf r}|R^{-1}\hat VR|\varphi}
=\braket{{\bf r}|R^{\dagger}\hat VR|\varphi}
=\int\braket{{\bf r}|R^{\dagger}\hat VR|{\bf r'}}\braket{{\bf r'}|\varphi}\d^3r'
=
$$
$$
=\int\braket{R{\bf r}|\hat V|R{\bf r'}}\braket{{\bf r'}|\varphi}\d^3r'
=\int V(R{\bf r},R{\bf r'})\varphi({\bf r'})\d^3r'
$$
where $R$ is the rotation operator (it's unitary). We have thus derived
$V(R{\bf r},R{\bf r'})=V({\bf r},{\bf r'})$ true for any $R$, which means that
the
the kernel only depends on $\rho$, $\rho'$ and 
${\bf\hat r}\cdot{\bf\hat r'}$, where ${\bf r}=\rho{\bf\hat r}$ and
${\bf r'}=\rho'{\bf\hat r'}$. So we obtain using \rno{fylm}
$$V({\bf r}, {\bf r'})=V(\rho,\rho',{\bf\hat r}\cdot{\bf\hat r'})=
\sum_{lm}  Y_{lm}({\bf\hat r}) V_l(\rho,\rho')
Y_{lm}^*({\bf\hat r'})$$
where
$$V_l(\rho,\rho')={(2l+1)^2\over8\pi}\int_{-1}^1 P_l(x)V_l(\rho,\rho',x)\d x$$
In Dirac notation:
$$V({\bf r}, {\bf r'})=\braket{{\bf r}|\hat V|{\bf r'}}
=\bra{\bf\hat r}\braket{\rho|\hat V|\rho'}\ket{\bf\hat r'}
=\sum_{lml'm'}\braket{{\bf\hat r}|lm}\bra{lm}\braket{\rho|\hat V|\rho'}
\ket{l'm'}\braket{l'm'|\bf\hat r'}
$$
From the above derivation we see that we must have:
$$\bra{lm}\braket{\rho|\hat V|\rho'}\ket{l'm'}=
V_l(\rho,\rho')\delta_{ll'}\delta_{mm'}$$
in other words
$$V_l(\rho,\rho')=\bra{lm}\braket{\rho|\hat V|\rho'}\ket{lm} \no{vlm2}$$
so we get
$$\braket{{\bf r}|\hat V|{\bf r'}}
=\sum_{lm}\braket{{\bf\hat r}|lm}V_l(\rho,\rho')\braket{lm|\bf\hat r'}
=\sum_{lm}Y_{lm}(\theta,\phi) V_l(\rho,\rho') Y_{lm}^*(\theta',\phi')
$$
and
$$\hat V 
=\sum_{lm}\ket{lm}\braket{lm|\hat V|lm}\bra{lm} 
=\sum_{lm}\ket{lm}\hat V_l\bra{lm} 
$$
where the operator $\hat V_l=\braket{lm|\hat V|lm}$ only acts on the radial
part of the wavefunction and according to \rno{vlm2} it doesn't depend on
$m$. Also according to \rno{vlm2} its matrix elements are
$$\braket{\rho|\hat V_l|\rho'}=\bra{lm}\braket{\rho|\hat V|\rho'}\ket{lm}=
V_l(\rho,\rho')$$

\section{Nonlocal Pseudopotentials}

A nonlocal pseudopotential $\hat V$ is just a general Hermitian operator.
We only want to construct pseudopotentials in the spherical problem, so every
pseudopotential can be written in the form \rno{lmexpansion}.
In practice we only use either {\it local\/} (the operator $\hat V$ is local)
or {\it semilocal\/} (the operator $\hat V$ is radially local, but angularly
nonlocal) pseudopotential.

Local potential (radially and angularly local) is defined by:
$$\braket{{\bf r}|\hat V|{\bf r'}}=V(\rho)\braket{{\bf r}|{\bf r'}}$$
so we can simply write
$$\hat V=V(\rho)\no{loc1}$$
so
$$V_l(\rho,\rho')
=\bra{lm}\braket{\rho|\hat V|\rho'}\ket{lm}
=V(\rho)\braket{\rho|\rho'}
=V(\rho){\delta(\rho-\rho')\over\rho^2}$$
so it turned out that the kernel is local doesn't depend on $l$ and we get
$$V({\bf r}(\rho,\theta,\phi), {\bf r'}(\rho',\theta',\phi'))=
\sum_{lm}Y_{lm}(\theta,\phi) V(\rho){\delta(\rho-\rho')\over\rho^2} Y_{lm}^*(\theta',\phi')=
$$
$$
=V(\rho){1\over\rho^2\sin\theta}
\delta(\rho-\rho')\delta(\theta-\theta')\delta(\phi-\phi')=
V(\rho)\delta({\bf r}-{\bf r}')$$
and
$$\braket{{\bf r}|\hat V|\varphi}=\int 
V(\rho)\delta({\bf r}-{\bf r}')
\varphi({\bf r'})\d^3r'=V(\rho)\varphi({\bf r})$$
so we recover \rno{loc1}. But we are just fooling around, there's nothing new in these formulas.

For a semilocal potential (radially local, but angularly nonlocal), the kernel
cannot depend on $m$ and is radially local, so:
$$\braket{\rho|\hat V_l|\rho'}=V_l(\rho,\rho')
=\bra{lm}\braket{\rho|\hat V|\rho'}\ket{lm}
=V_l(\rho)\braket{\rho|\rho'}
=V_l(\rho){\delta(\rho-\rho')\over\rho^2}$$
so the kernel is local and does depend on $l$ and we simply write
$$\hat V_l=V_l(\rho)$$
and
$$\hat V=\sum_{lm} \ket{lm}V_l(\rho)\bra{lm}\no{semi}$$
We can also calculate the same result explicitly in the $\bf r$
representation:
$$V({\bf r}(\rho,\theta,\phi), {\bf r'}(\rho',\theta',\phi'))=
\sum_{lm}Y_{lm}(\theta,\phi) V_l(\rho){\delta(\rho-\rho')\over\rho^2} Y_{lm}^*(\theta',\phi')
$$
and
$$\braket{{\bf r}|\hat V|\varphi}=\int 
\sum_{lm}Y_{lm}(\theta,\phi) V_l(\rho){\delta(\rho-\rho')\over\rho^2} Y_{lm}^*(\theta',\phi')
\varphi({\bf r'})\d^3r'=$$
$$=
\sum_{lm}Y_{lm}(\theta,\phi) V_l(\rho) \int Y_{lm}^*(\theta',\phi')
\varphi(\rho{\bf\hat r'})\d\Omega'
$$
or in Dirac notation
$$\braket{{\bf r}|\hat V|\varphi}=
\sum_{lm} \braket{{\bf\hat r}|lm}V_l(\rho) \bra{lm}\braket{\rho|\varphi}
$$
and we recover \rno{semi}.

So, to sum it up: semilocal pseudopotential is a general hermitian operator
in the spherically symmetric problem (i.e. $\hat V=R^{-1}\hat VR$) and radially
local. All such operators can be written in the form \rno{semi}.

Now, it can be shown that if we make the assumption of radial locality, we get
"correct" wavefunctions and energies in the linear approximation. We generally
only take a few terms in the expansion \rno{semi}, usually only $V_0$,
$V_1$ and $V_2$, sometimes also $V_3$ and $V_4$.

\section{Separable Potentials}

The pseudopotential above (Hamman, Schl\"uter, Chiang) has the form
$$\hat V=\sum_{lm} \ket{lm}V_l(\rho)\bra{lm}
=V_{loc}(\rho)+\sum_{lm} \ket{lm}[V_l(\rho)-V_{loc}(\rho)]\bra{lm}$$
Or, equivalently, in the $\bf r$ representation:
$$V({\bf r},{\bf r'})=\braket{{\bf r}|\hat V|{\bf r'}}=
V_{loc}(\rho)\delta({\bf r}-{\bf r'})+{\delta(\rho-\rho')\over\rho^2}
\sum_{lm}Y_{lm}({\bf\hat r})[V_l(\rho)-V_{loc}(\rho)]Y_{lm}^*({\bf\hat r'})$$
The first term doesn't cause a problem. Let's denote the second term (which is
semilocal) simply by $v$:
$$v=\sum_{lm} \ket{lm}[V_l(\rho)-V_{loc}(\rho)]\bra{lm}$$
Let's choose a complete but otherwise arbitrary set of functions
$\ket{\phi_i}$ (they contain both a radial and an angular dependence) and
define a matrix $U$ is by the equation
$$\sum_j U_{ij}\braket{\phi_j|v|\phi_k}=\delta_{ik}$$
then ($\ket{\psi}=\ket{\phi_k}\alpha_k$):
$$v\ket{\psi}
=\sum_{ik}v\ket{\phi_i}\delta_{ik}\alpha_k
=\sum_{ijk}v\ket{\phi_i}U_{ij}\braket{\phi_j|v|\phi_k}\alpha_k
=\sum_{ij}v\ket{\phi_i}U_{ij}\braket{\phi_j|v|\psi}
$$
So any Hermitian operator (including $v$) can be transformed exactly into the
following form
$$v=\sum_{ij}v\ket{\phi_i}U_{ij}\bra{\phi_j}v$$
We diagonalize the matrix $U$ by choosing
such functions $\ket{\bar\phi_i}$ for which the matrix
$\braket{\bar\phi_j|v|\bar\phi_k}$ (and hence the corresponding matrix $U$) is
equal to \one. We can find such functions for example using the Gram-Schmidt
orthogonalization procedure on $\ket{\phi_i}$ with a norm $\braket{f|v|g}$ (for
functions $f$ and $g$), more on that later. Then 
$$v
=\sum_{i}v\ket{\bar\phi_i}{1\over\braket{\bar\phi_i|v|\bar\phi_i}}
\bra{\bar\phi_i}v 
=\sum_{i}v\ket{\bar\phi_i}\bra{\bar\phi_i}v 
\no{vsep}$$
We could take any $\ket{\phi_i}$ and orthogonalize them. But because we have
$v$ in the form of \rno{semi}, we will be using $\ket{\phi_i}$ in the form
$\ket{\phi_i}=\ket{R_{nl}}\ket{lm}$, because it turns out we will only need to orthogonalize the radial
parts. The first term in \rno{vsep} then corresponds to
the KB potential. We of course take more terms and get accurate results without
ghost states.

Let's look at the orthogonalization. We start with the 
wavefunctions:
$$\ket{\phi_i}=\ket{R_{nl}}\ket{lm}$$
where $R_{nl}(\rho)=\braket{\rho|R_{nl}}$ and $i$ goes over all possible
triplets $(nlm)$, for example in this order (but any order is fine):

\vbox{\hfil\vbox{
\halign{
\hfill$#$\quad & $#$\quad & $#$\quad & \hfill$#$\cr
i & n & l & m \cr
\noalign{\smallskip \hrule \smallskip}
1 & 1 & 0 & 0 \cr
2 & 2 & 0 & 0 \cr
3 & 2 & 1 & -1 \cr
4 & 2 & 1 & 0 \cr
5 & 2 & 1 & 1 \cr
6 & 3 & 0 & 0 \cr
7 & 3 & 1 & -1 \cr
8 & 3 & 1 & 0 \cr
9 & 3 & 1 & 1 \cr
10 & 3 & 2 & -2 \cr
11 & 3 & 2 & -1 \cr
\noalign{\hbox{\quad\quad\dots}}
}
}\hfil}

We can also relate the $i$ and $n$, $l$, $m$ using this formula
$$i_{nlm}=\sum_{k=1}^{n-1}k^2+\left(\sum_{k=0}^{l-1} (2k+1)\right) + (l+m+1)=
{(n-1)n(2n-1)\over6} + l(l+1)+m+1$$

The operator $v$ acts on these $\ket{\phi_i}$ like this
$$
\braket{{\bf r}|v|\phi_i}
=\braket{{\bf r}|v|R_{nl}}\ket{lm}
=\bra{{\bf\hat r}}\braket{\rho|V_l(\rho)|R_{nl}}\ket{lm}
=V_l(\rho)R_{nl}(\rho)Y_{lm}({\bf\hat r})$$
Now we need to construct new orthogonal set of functions $\ket{\bar\phi_i}$ 
satisfying
$$\braket{\bar\phi_i|v|\bar\phi_j}=\delta_{ij}$$
This can be done using several methods, we chose the Gram-Schmidt
orthogonalization procedure, which works according to the following scheme:
$$\eqalign{
\ket{\tilde\phi_1}&=\one{1\over\sqrt{\braket{\phi_1|v|\phi_1}}}\ket{\phi_1}
;\quad\quad 
\quad\quad\quad\quad\quad
\quad\quad\quad\quad\quad
 \ket{\bar\phi_1}={1\over\sqrt{\braket{\tilde\phi_1|v|\tilde\phi_1}}}
 \ket{\tilde\phi_1}\cr
\ket{\tilde\phi_2}&=
\left(\one
-\ket{\bar\phi_1}\bra{\bar\phi_1}v
\right){1\over\sqrt{\braket{\phi_2|v|\phi_2}}}\ket{\phi_2};\quad\quad
\quad\quad\quad\quad\quad
 \ket{\bar\phi_2}={1\over\sqrt{\braket{\tilde\phi_2|v|\tilde\phi_2}}}
 \ket{\tilde\phi_2}\cr
\ket{\tilde\phi_3}&=
\left(\one
-\ket{\bar\phi_1}\bra{\bar\phi_1}v
-\ket{\bar\phi_2}\bra{\bar\phi_2}v
\right){1\over\sqrt{\braket{\phi_3|v|\phi_3}}}\ket{\phi_3};\quad\quad
 \ket{\bar\phi_3}={1\over\sqrt{\braket{\tilde\phi_3|v|\tilde\phi_3}}}
 \ket{\tilde\phi_3}\cr
\dots&\cr
}$$
We can verify by a direct calculation that this procedure ensures
$$\braket{\bar\phi_i|v|\bar\phi_j}=\delta_{ij}$$
It may be useful to compute the normalization factors explicitly:
$$\eqalign{
\braket{\tilde\phi_1|v|\tilde\phi_1}&=1\cr
\braket{\tilde\phi_2|v|\tilde\phi_2}&=1
  -{\braket{\phi_2|v|\bar\phi_1}\braket{\bar\phi_1|v|\phi_2}
  \over\braket{\phi_2|v|\phi_2}}\cr
\braket{\tilde\phi_3|v|\tilde\phi_3}&=1
  -{\braket{\phi_3|v|\bar\phi_1}\braket{\bar\phi_1|v|\phi_3}+
    \braket{\phi_3|v|\bar\phi_2}\braket{\bar\phi_2|v|\phi_3}
  \over\braket{\phi_3|v|\phi_3}}\cr
...&
}$$
we can also write down a first few orthogonal vectors explicitly:
$$\eqalign{
\ket{\bar\phi_1}&={\ket{\phi_1}\over\sqrt{\braket{\phi_1|v|\phi_1}}} \cr
\ket{\bar\phi_2}&={\ket{\phi_2}\braket{\phi_1|v|\phi_1}-\ket{\phi_1}\braket{\phi_1|v|\phi_2}
\over\sqrt{(\braket{\phi_1|v|\phi_1}\braket{\phi_2|v|\phi_2}-\braket{\phi_2|v|\phi_1}\braket{\phi_1|v|\phi_2})\braket{\phi_1|v|\phi_1}\braket{\phi_2|v|\phi_2}}} \cr
}$$
Now the crucial observation is
$$\bra{lm}\braket{R_{nl}|v|R_{n'l'}}\ket{l'm'}=
\braket{R_{nl}|V_l(\rho)|R_{n'l}}\delta_{ll'}\delta_{mm'} $$
which means that $\braket{\phi_i|v|\phi_j}=0$ if $\ket{\phi_i}$ and
$\ket{\phi_j}$ have different $l$ or $m$. In other words
$\ket{\phi_i}$ and $\ket{\phi_j}$ for different $\ket{lm}$ are already
orthogonal.
Thus the G-S orthogonalization
procedure only makes the $R_{nl}$ orthogonal for the same $\ket{lm}$. To get
explicit expressions for $\ket{\bar\phi_i}$, we simply use the formulas above
and get:
$$\ket{\phi_i}=\ket{R_{nl}}\ket{lm}\quad\to\quad
\ket{\bar\phi_i}=\ket{\bar R_{nl}}\ket{lm}
$$
where we have constructed new $\ket{\bar R_{nl}}$ from original $\ket{R_{nl}}$:
$$\eqalign{
\ket{\bar R_{10}}&={\ket{R_{10}}\over\sqrt{\braket{R_{10}|V_0|R_{10}}}}\cr
\ket{\bar R_{20}}&={\ket{R_{20}}
  -\ket{\bar R_{10}}\braket{\bar R_{10}|V_0|R_{20}}\over\sqrt{\dots}}\cr
\ket{\bar R_{21}}&={\ket{R_{21}}\over\sqrt{\braket{R_{21}|V_1|R_{21}}}}\cr
\ket{\bar R_{30}}&={\ket{R_{30}}
  -\ket{\bar R_{10}}\braket{\bar R_{10}|V_0|R_{30}}
  -\ket{\bar R_{20}}\braket{\bar R_{20}|V_0|R_{30}}
  \over\sqrt{\dots}}\cr
\ket{\bar R_{31}}&={\ket{R_{31}}
  -\ket{\bar R_{21}}\braket{\bar R_{21}|V_1|R_{31}}
  \over\sqrt{\dots}}\cr
\ket{\bar R_{32}}&={\ket{R_{32}}\over\sqrt{\braket{R_{32}|V_1|R_{32}}}}\cr
\ket{\bar R_{40}}&={\ket{R_{40}}
  -\ket{\bar R_{10}}\braket{\bar R_{10}|V_0|R_{40}}
  -\ket{\bar R_{20}}\braket{\bar R_{20}|V_0|R_{40}}
  -\ket{\bar R_{30}}\braket{\bar R_{30}|V_0|R_{40}}
  \over\sqrt{\dots}}\cr
\ket{\bar R_{41}}&={\ket{R_{41}}
  -\ket{\bar R_{21}}\braket{\bar R_{21}|V_1|R_{41}}
  -\ket{\bar R_{31}}\braket{\bar R_{31}|V_1|R_{41}}
  \over\sqrt{\dots}}\cr
&\dots\cr
}$$
Ok, so we have constructed new $\ket{\bar R_{nl}}$ from $\ket{R_{nl}}$ which
obey
$$\braket{\bar R_{nl}|V_l|\bar R_{n'l}}=\delta_{nn'}\no{orthog}$$
so for every $V_l$, we construct $\ket{\bar R_{nl}}$ for $n=l+1,\,\, l+2,
\cdots$.
Let's
continue:
$$v\ket{\bar\phi_i}=V_l(\rho)\ket{\bar R_{nl}}\ket{lm}$$
and finally we arrive at the separable form of the $l$ dependent
pseudopotential
$$v
=\sum_{i}v\ket{\bar\phi_i}\bra{\bar\phi_i}v 
=\sum_{i}V_l(\rho)\ket{\bar R_{nl}}\ket{lm}
\bra{lm}\bra{\bar R_{nl}}V_l(\rho)\no{Vsep}
$$
Note: the $V_l$ is actually $V_l-V_{loc}$, but this is just a detail.

To have some explicit formula, let's write how the separable potential acts on
a wavefunction:
$$(v\psi)({\bf r})=\braket{{\bf r}|v|\psi}= 
\sum_i\braket{{\bf\hat r}|lm}\braket{\rho|V_l(\rho)|\bar R_{nl}}
\bra{\bar R_{nl}}V_l(\rho)\braket{lm|\psi}=$$
$$
=\sum_iY_{lm}({\bf\hat r})\bar R_{nl}(\rho)V_l(\rho)
\int \bar R_{nl}(\rho')V_l(\rho')\int Y_{lm}^*({\bf\hat r'})\psi({\bf r'})\,\d
\Omega'\,\rho'^2 \d\rho'=$$
$$
=\sum_iY_{lm}({\bf\hat r})\bar R_{nl}(\rho)V_l(\rho)
\int \bar R_{nl}(\rho')V_l(\rho') Y_{lm}^*({\bf\hat r'})\psi({\bf r'})\,\d^3r' 
$$

To have some insight on what we are actually doing: we are making the local
potential $V_l$ nonlocal using:
$$V_l=\sum_{n=l+1}^\infty V_l\ket{\bar R_{nl}}\bra{\bar R_{nl}}V_l \no{Vlsep}$$
where
$$\braket{\bar R_{nl}|V_l|\bar R_{n'l}}=\delta_{nn'}$$
or in ${\bf r}$ representation:
$$V_l(\rho)\psi(\rho {\bf\hat r})=\sum_n V_l(\rho)\bar R_{nl}(\rho)
\int\bar R_{nl}(\rho')V_l(\rho')\psi(\rho'{\bf\hat r})\rho'^2\d \rho'$$
which is useful when computing integrals of this type
$$V_{ij}
=\int\phi_i(\rho) V_l \phi_j(\rho) \rho^2\d^3 \rho
=\braket{i|V_l|j}=
\sum_n \braket{i|V_l|{\bar R_{nl}}}\braket{\bar R_{nl}|V_l|j}
$$
$$\braket{i|V_l|{\bar R_{nl}}}=\int\phi_i(\rho)V_l(\rho)\bar R_{nl}(\rho)
\rho^2\d\rho$$
because the integral on the left hand side actually represents $N^2$ integrals,
where $N$ is the number of basis vectors $\ket{\phi_i}$. The sum on the
right hand side however only represents $K\cdot N$ integrals, where $K$ is the
number of terms taken into account in \rno{Vlsep}. Of course taking only finite
number of terms in \rno{Vlsep} is only an approximation to $\hat V_l$. In our
case, we don't need these 1D integrals (which can be easily computed directly,
because $V_l$ is local and the basis functions $\phi_i$ are nonzero only around
a node in the mesh, which means that the matrix $V_{ij}$ is sparse), but 3D
integrals, where angular parts of $V$ are nonlocal and radial part is local (so
the matrix $V_{ij}$ is dense), so the above procedure is the only way how to
proceed, because we decompose the matrix $V_{ij}$ into the sum of matrices in
the form $p_ip_j^*$, which can easily be handled and solved.

The scheme for the separation described
above works for any functions $R_{nl}(\rho)$. 
Because of the form of the expansion
\rno{Vlsep} however, we will use $R_{nl}$ from one atomic calculation.
We need to approximate $V_l$ by as few terms as
possible, 
so imagine how the $V_l(\rho)$ acts on the lowest radial function in the $l$
subspace, which is $\ket{R_{l+1;l}}$ and we see that all the terms in
\rno{Vlsep} except the first one 
$V_l\ket{\bar R_{l+1;l}}\bra{\bar R_{l+1;l}}V_l$
give zero, because they are orthogonal to $\ket{R_{l+1;l}}$. For the function
$\ket{R_{l+2;l}}$ all the terms except the first two are zero, because
$\braket{\bar R_{nl}|V_0|R_{l+2;l}}\neq0$ only for $n=l+1$ or $n=l+2$ 
(because the vectors
$\ket{R_{l+1;l}}$ and $\ket{R_{l+2;l}}$ span the same subspace as
$\ket{\bar R_{l+1;l}}$ and $\ket{\bar R_{l+2;l}}$ and using \rno{orthog})
For functions, which are a little different from all $\ket{R_{nl}}$ ($n>l$),
we won't genereally get precise results taking any (finite) number of terms
in \rno{Vlsep}, but the higher terms should give smaller and smaller
corrections.

So, to sum it up: We take all the $V_l$ in \rno{Vsep}  as we did in \rno{semi}.
Theoretically we should take $\bar R_{nl}$ for all 
$n=l+1,\,\, l+2,\,\,l+3,\dots$, 
but practically it suffices to only take several $\bar R_{nl}$ for a given $l$
from one atomic calculaction.

Let's give an example: we are calculating 14 electrons, so we will only
take into account the lowest 14 eigenvalues in the Kohn sham equations, which
are $\ket{\phi_1}$ up to $\ket{\phi_{14}}$. The lowest radial functions in each
$l$ subspace are $\ket{\phi_i}$ for $i=1,3,4,5,10,11,12,13,14$ and on these 9
functions we get a precise result with only one term in the expansion
\rno{Vlsep}. For the other 5 functions ($i=2,6,7,8,9$) we will have to take into
account more terms. Let's look in more detail at the case $l=0$ (i.e.
$i=1,2,6$). Then 
$$V_0=
V_0\ket{\bar R_{10}}\bra{\bar R_{10}}V_0+
V_0\ket{\bar R_{20}}\bra{\bar R_{20}}V_0+
V_0\ket{\bar R_{30}}\bra{\bar R_{30}}V_0+
\dots
$$
and for the case $i=1$ we see that one term in \rno{Vlsep} is enough:
$$v\ket{\phi_1}=v\ket{R_{10}}\ket{00}=V_0\ket{R_{10}}\ket{00}=
V_0\ket{\bar R_{10}}\bra{\bar R_{10}}V_0\ket{R_{10}}\ket{00}
$$
because $\braket{\bar R_{n0}|V_0|R_{10}}=0$ for $n>1$.
For the case $i=2$ we get the correct result with 2 terms in \rno{Vlsep}
$$v\ket{\phi_2}=v\ket{R_{20}}\ket{00}=V_0\ket{R_{20}}\ket{00}=(
V_0\ket{\bar R_{10}}\bra{\bar R_{10}}V_0\ket{R_{20}}+
V_0\ket{\bar R_{20}}\bra{\bar R_{20}}V_0\ket{R_{20}}
)\ket{00}
$$
because $\braket{\bar R_{n0}|V_0|R_{20}}=0$ for $n>2$.
For the case $i=6$ we need to take into account 3 terms etc. 
We can see from this example, that taking $\ket{R_{nl}}$ from one atomic
calculation, we get precise results (with the same atom) only taking into
account a finite number of terms in \rno{Vlsep}, for 14 electrons actually only
3 terms. For several atoms calculation, we won't get precise results, but it
should be a sufficiently good approximation.

The described method is general, the only drawback is that if we don't take
functions $\ket{R_{nl}}$ which are similar to the solution, we need to take a
lot of terms in \rno{Vlsep}, resulting in many matrices of the form $p_ip_j^*$,
which we don't want, even though, theoretically we can get a solution with any
precision we want taking more and more terms in \rno{Vlsep}.

See also \cite{blochl}.
