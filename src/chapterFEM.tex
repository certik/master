\chapter{FEM}

\section{Introduction}

This chapter explains FEM and gives concrete formulas which are needed in the
calculation.

\section{Weak Formulation of the Schr\"odinger Equation}

One particle Schr\"odinger equation is
$$\left(-{\hbar^2\over2m}\nabla^2 + V\right)\psi=E\psi\,.$$
We multiply both sides by a test function $v$
$$-\left({\hbar^2\over2m}\nabla^2\psi\right)v=(E-V)\psi v\,,$$
and integrate over the whole volume we are interested in
$$\int-\left({\hbar^2\over2m}\nabla^2\psi\right)v\,\d V=\int(E-V)\psi v\,\d
V\,,\no{1}$$
and using the vector identity
$$-\left(\nabla^2\psi)\right)v=\nabla \psi\cdot
\nabla v - \nabla\cdot\left((\nabla \psi)v\right),$$
we rewrite the left hand side of \rno{1}
$$\int{\hbar^2\over2m}\nabla\psi\cdot\nabla v\,\d V=\int(E-V)\psi v\,\d
V+\int{\hbar^2\over2m}\nabla\cdot\left((\nabla \psi)v\right)\,\d V\,,$$
now we apply Gauss Theorem
$$\int{\hbar^2\over2m}\nabla\psi\cdot\nabla v\,\d V=\int(E-V)\psi v\,\d
V+\oint{\hbar^2\over2m}(\nabla \psi)v\cdot{\bf n}\,\d S\,,$$
and rewriting $\nabla\psi\cdot{\bf n}\equiv{\d\psi\over\d n}$
%$$\int{\hbar^2\over2m}\nabla\psi\cdot\nabla v\,\d V=\int(E-V)\psi v\,\d
%V+\oint{\hbar^2\over2m}{\d\psi\over\d n}v\,\d S\,,\no{w}$$
$$\int{\hbar^2\over2m}\nabla\psi\cdot\nabla v\,\d V+ \int vV\psi\,\d V
=
\int E\psi v\,\d V + \oint{\hbar^2\over2m}{\d\psi\over\d n}v\,\d S\,,\no{w}$$
which is the weak formulation. The problem reads: find a function $\psi$ such
that \rno{w} holds for every $v$.

\section{Finite Elements}

We choose a basis $\phi_i$ and substitute $\phi_i$ for $v$ and expand
$\psi=\sum q_j\phi_j$
$$\left(\int{\hbar^2\over2m}\nabla\phi_j\cdot\nabla\phi_i\,\d V+
\int\phi_iV\phi_j\,\d V\right)q_j
=
\left(\int E\phi_j\phi_i\,\d V\right)q_j
+\oint{\hbar^2\over2m}{\d\psi\over\d n}\phi_i\,\d S\,,\no{fem}$$
which can be written in a matrix form
$$\left(K_{ij}+V_{ij}\right)q_j=EM_{ij}q_j+F_i\,,$$
where
$$\eqalign{
V_{ij}&=\int\phi_iV\phi_j\,\d V\,,\cr
M_{ij}&=\int\phi_i\phi_j\,\d V\,,\cr
K_{ij}&={\hbar^2\over2m}\int\nabla\phi_i\cdot\nabla\phi_j\,\d V\,,\cr
F_i&={\hbar^2\over2m}\oint{\d\psi\over\d n}\phi_i\,\d S\,.\cr
}$$
Usually we set $F_i=0$.

We decompose the domain into elements and compute the integrals as the sum over
elements. For example:
$$K_{ij}=\sum_{E\in elements} K_{ij}^E$$
where $K_{ij}^E$ is the integral over one element only
$$
K_{ij}^{E}=\int{\hbar^2\over2m}\nabla\phi_j\cdot\nabla\phi_i\,\d V^{E}\approx
\sum_{q=0}^{N_q-1}{\hbar^2\over2m}\,\nabla\phi_i(x_q)\cdot\nabla\phi_j(x_q)\,
w_q|\det J(\hat x_q)|\,.
$$
The integral is computed numerically using a Gauss integration: $x_q$ are Gauss
points (there are $N_q$ of them), $w_q$ is the weight of each point, and the
Jacobian $|\det J(\hat x_q)|$ is there because we are actually computing the
integral on the reference element instead in the real space.

The surface integrals are computed similarly.

\section{Pseudopotentials Formulation}

There are no problems with other matrix elements in \rno{fem} except 
$$V_{ij}
=\int\phi_i({\bf r}) V \phi_j({\bf r}) \d^3 r
=\int\braket{i|{\bf r}} \braket{{\bf r}|\hat V|j} \d^3 r
=\braket{i|\hat V|j}
$$
where 
$$\hat V=V_{loc}(\rho)+\sum_{nlm}V_l(\rho)\ket{\bar R_{nl}}\ket{lm}
\bra{lm}\bra{\bar R_{nl}}V_l(\rho)$$
so
$$
V_{ij}=\braket{i|V_{loc}(\rho)|j}+
\bra{i}\sum_{nlm}V_l(\rho)\ket{\bar R_{nl}}\ket{lm}
\bra{lm}\bra{\bar R_{nl}}V_l(\rho) \ket{j}
=V_{ij}^{loc}+\sum_{nlm}p_ip_j^*
$$
where the complex vector $p_i$ is given by
$$p_i^{(nlm)}=\braket{i|lm}V_l(\rho)\ket{\bar R_{nl}}=
\int\braket{i|{\bf r}}\braket{{\bf\hat r}|lm}
\braket{\rho|V_l(\rho)|\bar R_{nl}}\d^3 r=
\int \phi_i({\bf r})Y_{lm}({\bf\hat r})V_l(\rho)\bar R_{nl}(\rho)\d^3 r
$$
and
$$V_{ij}^{loc}=\int\phi_i({\bf r}) V_{loc}(\rho) \phi_j({\bf r}) \d^3 r $$
and
$Y_{lm}({\bf\hat r})$, $\bar R_{nl}(\rho)$ and $V_l(\rho)$ are given functions.
Noticing that
$$\sum_m p_ip_j^*=
\int \phi_i({\bf r})Y_{lm}({\bf\hat r})V_l(\rho)\bar R_{nl}(\rho)\d^3 r
\int \phi_j({\bf r'})Y_{lm}^*({\bf\hat r'})V_l(\rho')\bar R_{nl}(\rho')\d^3 r'=
$$
$$
=\int\int Y_{lm}({\bf\hat r})Y_{lm}^*({\bf\hat r'})V_l(\rho)\bar R_{nl}(\rho)
\phi_i({\bf r}) \phi_j({\bf r'})V_l(\rho')\bar R_{nl}(\rho')\d^3 r\d^3 r'=
$$
and using \rno{lsum} we get
$$\sum_m p_ip_j^*=
\int\int {4\pi\over 2l+1}P_l({\bf\hat r}\cdot{\bf\hat r'})V_l(\rho)\bar
R_{nl}(\rho) \phi_i({\bf r}) \phi_j({\bf r'})V_l(\rho')\bar R_{nl}(\rho')\d^3
r\d^3 r'
$$
which is a real number, thus $\sum_{nlm}p_ip_j^*$ is also a real number, which
means that we can calculate with only the real parts of the matrix $p_ip_j^*$,
because the imaginary parts cancels out in the result:
$$\sum_{nlm}p_ip_j^*=\Re\left(\sum_{nlm}p_ip_j^*\right)=
\sum_{nlm}\Re(p_ip_j^*)$$
let $p_i=a_i+ib_i$ then 
$$\Re(p_ip_j^*)=\Re((a_i+ib_i)(a_j-ib_j))=a_ia_j+b_ib_j$$
and
$$V_{ij}=V_{ij}^{loc}+\sum_{nlm}(a_ia_j+b_ib_j)$$
where
$$a_i=
\sqrt{{2l+1\over4\pi}{(l-m)!\over(l+m)!}}
\int \phi_i({\bf r})P_l^m(\cos\theta)\cos(m\phi)V_l(\rho)\bar
R_{nl}(\rho)\d^3 r
$$
$$b_i=
\sqrt{{2l+1\over4\pi}{(l-m)!\over(l+m)!}}
\int \phi_i({\bf r})P_l^m(\cos\theta)\sin(m\phi)V_l(\rho)\bar
R_{nl}(\rho)\d^3 r
$$
just don't confuse the basis function $\phi_i({\bf r})$ with the spherical
integration variable $\phi$.

\section{Example on Si}

There are only two valence electrons to take into account, thus we have only 2
$n$ in the summation $(nlm)$. The potentials in the Schr\"odinger equation are
$$V=V_{local}+V_{nonlocal}$$
$$V_{local}=V_H+V_{XC}+V_{loc}$$
$$\hat V_{nonlocal}=
\sum_{nlm}V_l(\rho)\ket{\bar R_{nl}}\ket{lm}
\bra{lm}\bra{\bar R_{nl}}V_l(\rho)$$
There are just $V_0$, $V_1$ and $V_2$.

\psfig{pseudopot.eps}{pseudofig}
{Local potential $V_{local}\equiv U$ and $l$-dependent pseudopotentials $V_l$}

We probably need to calculate
$\bar R_{10}$,
$\bar R_{20}$,
$\bar R_{21}$,
$\bar R_{30}$,
$\bar R_{31}$,
$\bar R_{32}$,
but I am not completely sure. We get maybe around 
18 complex matrices, which means 36 real matrices of the form
$a_ia_j$. The input for the solver is 36 real vectors $a$, $b$, $c$, $d$,
$e$,\dots and sparse matrices $V^{loc}$, $M$, and $K$. The solver needs to
solve
$$(K+V^{loc}+a^Ta+b^Tb+c^Tc+d^Td+e^Te+\cdots)q=EMq$$
