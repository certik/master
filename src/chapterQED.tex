\chapter{Derivation from QED}

\section{Introduction}

The correct theory for electrons is the quantum electrodynamics (QED). Thus all
the equations which we are going to use, follow from the QED and this chapter
is devoted for their derivation. The standard nonrelativistic quantum
mechanics (QM)
contains a lot of ad-hoc assumptions (Schr\"odinger equation, electromagnetic
coupling, spin, etc.), that cannot be satisfactorily explained. It is true that
the whole nonrelativistic QM can be derived from a small set of
assumptions (mostly commutation relations and some particular form of QM
operators like spin), but it is not at all clear why these assumptions look the
way they look. The QM is a set of rules which are known to work quite
well, but obviously, we are not interested in postulating correct forms of
operators or commutation relations, we want a better explanation, if there is
one. And fortunately, there is.

The correct theory so far is the Standard Model and in our case of electron
structure calculations, all phenomena are explained by the QED. There are still
some problems in the Standard Model, but for much higher energies than
we deal with in the electron structure. Thus what we do is that we take
the most complete theory so far (QED) and derive everything from it. Of course,
because we are dealing with low energies and to ease calculations, we make a
lot of approximations on our way, but this is the only correct way to
proceed, because it will be clear that we make each approximation because of
this and that. If it turns out our results are not in agreement with experiment
(for example the Schr\"odinger equation is imprecise), we need to neglect
less things. 

As you will see, this chapter is not long, so there really is no excuse of
doing things differently (maybe in 1940s, but certainly not now when we have
the QED). 



\def\L{L}

\section{QED}

The QED Lagrangian density is
$$\L=\bar\psi(ic\gamma^\mu D_\mu-mc^2)\psi-{1\over4}F_{\mu\nu}F^{\mu\nu}$$
where
$$\psi=\left(\matrix{\psi_1\cr\psi_2\cr\psi_3\cr\psi_4\cr}\right) $$
and
$$D_\mu=\partial_\mu+ieA_\mu$$
is the gauge covariant derivative and ($e$ is the elementary charge, which is
$1$ in atomic units)
$$F_{\mu\nu}=\partial_\mu A_\nu-\partial_\nu A_\mu$$
is the electromagnetic field tensor. It's astonishing, that this simple
Lagrangian can account for all phenomena from macroscopic scales down to
something like $10^{-13}\rm\,cm$. So of course Feynman, Schwinger and Tomonaga
received the 1965 Nobel Prize in Physics for such a fantastic achievement.

Plugging this Lagrangian into the Euler-Lagrange equation of motion for a
field, we get:
$$(ic\gamma^\mu D_\mu-mc^2)\psi=0$$
$$\partial_\nu F^{\nu\mu}=-ec\bar\psi\gamma^\mu\psi$$
The first equation is the Dirac equation in the electromagnetic field and
the second equation is a set of Maxwell equations ($\partial_\nu
F^{\nu\mu}=-ej^\mu$) with a source $j^\mu=c\bar\psi\gamma^\mu\psi$, which is a
4-current comming from the Dirac equation.

The fields $\psi$ and $A^\mu$ are quantized. The first approximation is that we
take $\psi$ as a wavefunction, that is, it is a classical 4-component field. It
can be shown that this corresponds to taking three orders in the perturbation
theory.

The first component $A_0$ of the 4-potential is the electric potential, and
because this is the potential that (as we show in a moment) is in the
Schr\"odinger equation, we denote it by $V$:
$$A_\mu=\left({V\over ec},A_1,A_2,A_3\right)$$
So in the non-relativistic limit, the $V\over e$ corresponds to the electric
potential.  We multiply the Dirac equation by $\gamma^0$ from left to get:
$$0=\gamma^0(ic\gamma^\mu D_\mu-mc^2)\psi=
\gamma^0(ic\gamma^0(\partial_0+i{V\over c})+ic\gamma^i
(\partial_i+ieA_i)-mc^2)\psi=$$
$$=
(ic\partial_0+ic\gamma^0\gamma^i\partial_i-\gamma^0mc^2-V
-ce\gamma^0\gamma^iA_i)\psi
$$
and we make the following substitutions (it's just a formalism, nothing more):
$\beta=\gamma^0$, $\alpha^i=\gamma^0\gamma^i$, $p_j=-i\partial_j$,
$\partial_0={1\over c}{\partial\over\partial t}$
to get
$$(i{\partial\over\partial t}-c\alpha^i p_i-\beta
mc^2-V-ce\alpha^iA_i)\psi=0\,.$$
This, in most solid state physics texts, is usually written as
$$ i{\partial\psi\over\partial t}=H\psi\,,$$
where the Hamiltonian is given by
$$ H=c\alpha^i(p_i+eA_i)+\beta mc^2+V\,.$$

The right hand side of the Maxwell equations is the 4-current, so it's given
by:
$$j^\mu=c\bar\psi\gamma^\mu\psi$$
Now me make the substitution $\psi=e^{-imt}\varphi$, which states, that we
separate the largest oscillations of the wavefunction and we get
$$j^0=c\bar\psi\gamma^0\psi=c\psi^\dagger\psi=c\varphi^\dagger\varphi$$
$$j^i=c\bar\psi\gamma^i\psi=c\psi^\dagger\alpha^i\psi=c\varphi^\dagger\alpha^i\varphi$$
The Dirac equation implies the Klein-Gordon equation:
$$(-ic\gamma^\mu D_\mu-mc^2)(ic\gamma^\nu D_\nu-mc^2)\psi=
(c^2\gamma^\mu\gamma^\nu D_\mu D_\mu+m^2c^4)\psi=
$$
$$
=(c^2D^\mu D_\mu-ic^2[\gamma^\mu,\gamma^\nu]D_\mu D_\nu+m^2c^4)\psi=0$$
Note however, the $\psi$ in the true Klein-Gordon equation is just a scalar,
but here we get a 4-component spinor. Now:
$$
D_\mu D_\nu = (\partial_\mu+ieA_\mu)(\partial_\nu+ieA_\nu)=
\partial_\mu\partial_\nu+ie(A_\mu\partial_\nu+A_\nu\partial_\mu+
(\partial_\mu A_\nu))-e^2A_\mu A_\nu
$$
$$[D_\mu, D_\nu] = D_\mu D_\nu-D_\nu D_\mu=ie(\partial_\mu A_\nu)-
ie(\partial_\nu A_\mu)$$
We rewrite $D^\mu D_\mu$:
$$D^\mu D_\mu=g^{\mu\nu}D_\mu D_\nu=
\partial^\mu\partial_\mu+ie((\partial^\mu
A_\mu)+2A^\mu\partial_\mu) -e^2A^\mu A_\mu=$$
$$=\partial^\mu\partial_\mu+
ie((\partial^0 A_0)+2A^0\partial_0+(\partial^i A_i)+2A^i\partial_i)
-e^2(A^0A_0+A^i A_i)=$$
$$=\partial^\mu\partial_\mu
+i{1\over c^2}{\partial V\over\partial t}+
2i{V\over c^2}{\partial\over\partial t}
+ie(\partial^i A_i)+2ieA^i\partial_i
 -{V^2\over c^2}-e^2A^iA_i$$

We use the identity
${\partial\over\partial t}\left(e^{-imt}f(t)\right)=
e^{-imt}(-im+{\partial\over\partial t})f(t)$ to get:

$$L=c^2\partial^\mu\psi^*\partial_\mu\psi-m^2c^4\psi^*\psi=
{\partial\over\partial t}\psi^*{\partial\over\partial t}\psi
-c^2\partial^i\psi^*\partial_i\psi-m^2c^4\psi^*\psi=$$
$$=(imc^2+{\partial\over\partial t})\varphi^*
(-imc^2+{\partial\over\partial t})\varphi
-c^2\partial^i\varphi^*\partial_i\varphi-m^2c^4\varphi^*\varphi=$$
$$=2mc^2\left[{1\over2}i(\varphi^*{\partial\varphi\over\partial t}-
\varphi{\partial\varphi^*\over\partial t})-
{1\over2m}\partial^i\varphi^*\partial_i\varphi
+{1\over2mc^2}{\partial\varphi^*\over\partial t}
{\partial\varphi\over\partial t}\right]$$
The constant factor $2mc^2$ in front of the Lagrangian is of course irrelevant,
so we drop it and then we take the limit $c\to\infty$ (neglecting the last
term) and we get
$$L={1\over2}i(\varphi^*{\partial\varphi\over\partial t}-
\varphi{\partial\varphi^*\over\partial t})-
{1\over2m}\partial^i\varphi^*\partial_i\varphi
$$
After integration by parts we arrive at
$$L=i\varphi^*{\partial\varphi\over\partial t}
-{1\over 2m}\partial^i\varphi^*\partial_i
\varphi$$
The nonrelativistic limit can also be applied directly to the Klein-Gordon equation:
$$0=(c^2D^\mu D_\mu+m^2c^4)\psi=$$
$$=\left(
c^2\partial^\mu\partial_\mu
+i{\partial V\over\partial t}
+2iV{\partial\over\partial t}
+iec^2(\partial^i A_i)
+2iec^2A^i\partial_i
-V^2
-e^2c^2A^iA_i
+m^2c^4
\right)e^{-imc^2t}\varphi=
$$
$$=\left(
{\partial^2\over\partial t^2}
-c^2\nabla^2
+2iV{\partial\over\partial t}
+i{\partial V\over\partial t}
+iec^2(\partial^i A_i)
+2iec^2A^i\partial_i
-V^2
-e^2c^2A^iA_i
+m^2c^4
\right)e^{-imc^2t}\varphi=
$$
$$=e^{-imc^2t}\left(
(-imc^2+{\partial\over\partial t})^2
-c^2\nabla^2
+2iV(-imc^2+{\partial\over\partial t})
+i{\partial V\over\partial t}
+iec^2(\partial^i A_i)
+2iec^2A^i\partial_i
-V^2+ \right.
$$
$$\left.
-e^2c^2A^iA_i
+m^2c^4
\right)\varphi=
$$
$$=e^{-imc^2t}\left(
-2imc^2{\partial\over\partial t}+{\partial^2\over\partial t^2}
-c^2\nabla^2
+2Vmc^2
+2iV{\partial\over\partial t}
+i{\partial V\over\partial t}
+iec^2(\partial^i A_i)
+2iec^2A^i\partial_i
-V^2+ \right.
$$
$$\left.
-e^2c^2A^iA_i
\right)\varphi=
$$
$$=
-2mc^2 e^{-imc^2 t} \left(i{\partial\over\partial t}+{\nabla^2\over2m}-V
-{1\over2mc^2}{\partial^2\over\partial t^2}-{i\over2mc^2}{\partial
V\over\partial t}+{V^2\over2mc^2}-{iV\over mc^2}{\partial\over\partial
t}+\right.$$
$$\left.-{ie\over2m}\partial^i A_i-{ie\over
m}A^i\partial_i+{e^2\over2m}A^iA_i\right)\varphi
$$
Taking the limit $c\to\infty$ we again recover the Schr\"odinger equation:
$$i{\partial\over\partial t}\varphi=\left(-{\nabla^2\over2
m}+V
+{ie\over2m}\partial^i A_i
+{ie\over m}A^i\partial_i
-{e^2\over2m}A^iA_i
\right)\varphi\,,$$
we rewrite the right hand side a little bit:
$$i{\partial\over\partial t}\varphi=\left({1\over2 m}
(\partial^i\partial_i
+ie\partial^i A_i
+2ieA^i\partial_i
-e^2A^iA_i
)
+V
\right)\varphi\,,$$
$$i{\partial\over\partial t}\varphi=\left({1\over2 m}
(\partial^i+ieA^i)(\partial_i+ieA_i)
+V
\right)\varphi\,,$$
And we get the usual form of the Schr\"odinger equation for the vector
potential ${\bf A}=(A_1, A_2, A_3)$:
$$i{\partial\over\partial t}\varphi=\left(-{(\nabla+ie{\bf A})^2\over2 m}
+V \right)\varphi\,.$$
