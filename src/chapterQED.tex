\chapter{Derivation from QED}

\section{Introduction}

The correct theory for electrons is the quantum electrodynamics (QED). Thus all
the equations which we are going to use, follow from the QED and this chapter
is devoted for their derivation. The standard nonrelativistic quantum
mechanics (QM)
contains a lot of ad-hoc assumptions (Schr\"odinger equation, electromagnetic
coupling, spin, etc.), that cannot be satisfactorily explained. It is true that
the whole nonrelativistic QM can be derived from a small set of
assumptions (mostly commutation relations and some particular form of QM
operators like spin), but it is not at all clear why these assumptions look the
way they look. The QM is a set of rules which are known to work quite
well, but obviously, we are not interested in postulating correct forms of
operators or commutation relations, we want a better explanation, if there is
one. And fortunately, there is.

The correct theory so far is the Standard Model and in our case of electron
structure calculations, all phenomena are explained by the QED. There are still
some problems in the Standard Model, but for much higher energies than
we deal with in the electron structure. Thus what we do is that we take
the most complete theory so far (QED) and derive everything from it. Of course,
because we are dealing with low energies and to ease calculations, we make a
lot of approximations on our way, but this is the only correct way to
proceed, because it will be clear that we make each approximation because of
this and that. If it turns out our results are not in agreement with experiment
(for example the Schr\"odinger equation is imprecise), we need to neglect
less things. 

As you will see, this chapter is not long, so there really is no excuse of
doing things differently (maybe in 1940s, but certainly not now when we have
the QED). 

There is one argument to be made though. As you will see, we will not get any
new equations by deriving everything from firt principles from QED. We
sometimes get a deeper understanding, however, from the practical point of
view, it is not necessary to do that. There are logical problems even in QED or
the Standard Model and noone really knows what are the correct equations,
because we simply don't have enough experiments. The higher the energy we go,
let's say up to the QED, the more fundamental theory we get, however, also the
less experiments we have and thus some logical steps are just plain hand
waving, because there is a whole bunch of ways how to continue, so we chose the
one which we like, or which is the mathematically simplest. Contrary, in the
low energy regime, like the classic quantum mechanics, the equations are very
well known and there is really no other option. On the other hand, the theory
doesn't explain us "why". So deriving everything from QED answers the "why",
but creates some more fundamental questions, but those needn't concern us,
because for our energy scale, everything works very nice.

\def\L{L}

\section{QED}

The QED Lagrangian density is
$$\L=\bar\psi(ic\gamma^\mu D_\mu-mc^2)\psi-{1\over4}F_{\mu\nu}F^{\mu\nu}$$
where
$$\psi=\left(\matrix{\psi_1\cr\psi_2\cr\psi_3\cr\psi_4\cr}\right) $$
and
$$D_\mu=\partial_\mu+ieA_\mu$$
is the gauge covariant derivative and ($e$ is the elementary charge, which is
$1$ in atomic units)
$$F_{\mu\nu}=\partial_\mu A_\nu-\partial_\nu A_\mu$$
is the electromagnetic field tensor. It's astonishing, that this simple
Lagrangian can account for all phenomena from macroscopic scales down to
something like $10^{-13}\rm\,cm$. So of course Feynman, Schwinger and Tomonaga
received the 1965 Nobel Prize in Physics for such a fantastic achievement.

Plugging this Lagrangian into the Euler-Lagrange equation of motion for a
field, we get:
$$(ic\gamma^\mu D_\mu-mc^2)\psi=0$$
$$\partial_\nu F^{\nu\mu}=-ec\bar\psi\gamma^\mu\psi$$
The first equation is the Dirac equation in the electromagnetic field and
the second equation is a set of Maxwell equations ($\partial_\nu
F^{\nu\mu}=-ej^\mu$) with a source $j^\mu=c\bar\psi\gamma^\mu\psi$, which is a
4-current comming from the Dirac equation.

The fields $\psi$ and $A^\mu$ are quantized. The first approximation is that we
take $\psi$ as a wavefunction, that is, it is a classical 4-component field. It
can be shown that this corresponds to taking three orders in the perturbation
theory.

The first component $A_0$ of the 4-potential is the electric potential, and
because this is the potential that (as we show in a moment) is in the
Schr\"odinger equation, we denote it by $V$:
$$A_\mu=\left({V\over ec},A_1,A_2,A_3\right)$$
So in the non-relativistic limit, the $V\over e$ corresponds to the electric
potential.  We multiply the Dirac equation by $\gamma^0$ from left to get:
$$0=\gamma^0(ic\gamma^\mu D_\mu-mc^2)\psi=
\gamma^0(ic\gamma^0(\partial_0+i{V\over c})+ic\gamma^i
(\partial_i+ieA_i)-mc^2)\psi=$$
$$=
(ic\partial_0+ic\gamma^0\gamma^i\partial_i-\gamma^0mc^2-V
-ce\gamma^0\gamma^iA_i)\psi
$$
and we make the following substitutions (it's just a formalism, nothing more):
$\beta=\gamma^0$, $\alpha^i=\gamma^0\gamma^i$, $p_j=-i\partial_j$,
$\partial_0={1\over c}{\partial\over\partial t}$
to get
$$(i{\partial\over\partial t}-c\alpha^i p_i-\beta
mc^2-V-ce\alpha^iA_i)\psi=0\,.$$
This, in most solid state physics texts, is usually written as
$$ i{\partial\psi\over\partial t}=H\psi\,,$$
where the Hamiltonian is given by
$$ H=c\alpha^i(p_i+eA_i)+\beta mc^2+V\,.$$

The right hand side of the Maxwell equations is the 4-current, so it's given
by:
$$j^\mu=c\bar\psi\gamma^\mu\psi$$
Now we make the substitution $\psi=e^{-imc^2t}\varphi$, which states, that we
separate the largest oscillations of the wavefunction and we get
$$j^0=c\bar\psi\gamma^0\psi=c\psi^\dagger\psi=c\varphi^\dagger\varphi$$
$$j^i=c\bar\psi\gamma^i\psi=c\psi^\dagger\alpha^i\psi=c\varphi^\dagger\alpha^i\varphi$$
The Dirac equation implies the Klein-Gordon equation:
$$(-ic\gamma^\mu D_\mu-mc^2)(ic\gamma^\nu D_\nu-mc^2)\psi=
(c^2\gamma^\mu\gamma^\nu D_\mu D_\mu+m^2c^4)\psi=
$$
$$
=(c^2D^\mu D_\mu-ic^2[\gamma^\mu,\gamma^\nu]D_\mu D_\nu+m^2c^4)\psi=0$$
Note however, the $\psi$ in the true Klein-Gordon equation is just a scalar,
but here we get a 4-component spinor. Now:
$$
D_\mu D_\nu = (\partial_\mu+ieA_\mu)(\partial_\nu+ieA_\nu)=
\partial_\mu\partial_\nu+ie(A_\mu\partial_\nu+A_\nu\partial_\mu+
(\partial_\mu A_\nu))-e^2A_\mu A_\nu
$$
$$[D_\mu, D_\nu] = D_\mu D_\nu-D_\nu D_\mu=ie(\partial_\mu A_\nu)-
ie(\partial_\nu A_\mu)$$
We rewrite $D^\mu D_\mu$:
$$D^\mu D_\mu=g^{\mu\nu}D_\mu D_\nu=
\partial^\mu\partial_\mu+ie((\partial^\mu
A_\mu)+2A^\mu\partial_\mu) -e^2A^\mu A_\mu=$$
$$=\partial^\mu\partial_\mu+
ie((\partial^0 A_0)+2A^0\partial_0+(\partial^i A_i)+2A^i\partial_i)
-e^2(A^0A_0+A^i A_i)=$$
$$=\partial^\mu\partial_\mu
+i{1\over c^2}{\partial V\over\partial t}+
2i{V\over c^2}{\partial\over\partial t}
+ie(\partial^i A_i)+2ieA^i\partial_i
 -{V^2\over c^2}-e^2A^iA_i$$

We use the identity
${\partial\over\partial t}\left(e^{-imc^2t}f(t)\right)=
e^{-imc^2t}(-imc^2+{\partial\over\partial t})f(t)$ to get:

$$L=c^2\partial^\mu\psi^*\partial_\mu\psi-m^2c^4\psi^*\psi=
{\partial\over\partial t}\psi^*{\partial\over\partial t}\psi
-c^2\partial^i\psi^*\partial_i\psi-m^2c^4\psi^*\psi=$$
$$=(imc^2+{\partial\over\partial t})\varphi^*
(-imc^2+{\partial\over\partial t})\varphi
-c^2\partial^i\varphi^*\partial_i\varphi-m^2c^4\varphi^*\varphi=$$
$$=2mc^2\left[{1\over2}i(\varphi^*{\partial\varphi\over\partial t}-
\varphi{\partial\varphi^*\over\partial t})-
{1\over2m}\partial^i\varphi^*\partial_i\varphi
+{1\over2mc^2}{\partial\varphi^*\over\partial t}
{\partial\varphi\over\partial t}\right]$$
The constant factor $2mc^2$ in front of the Lagrangian is of course irrelevant,
so we drop it and then we take the limit $c\to\infty$ (neglecting the last
term) and we get
$$L={1\over2}i(\varphi^*{\partial\varphi\over\partial t}-
\varphi{\partial\varphi^*\over\partial t})-
{1\over2m}\partial^i\varphi^*\partial_i\varphi
$$
After integration by parts we arrive at
$$L=i\varphi^*{\partial\varphi\over\partial t}
-{1\over 2m}\partial^i\varphi^*\partial_i
\varphi$$
The nonrelativistic limit can also be applied directly to the Klein-Gordon equation:
$$0=(c^2D^\mu D_\mu+m^2c^4)\psi=$$
$$=\left(
c^2\partial^\mu\partial_\mu
+i{\partial V\over\partial t}
+2iV{\partial\over\partial t}
+iec^2(\partial^i A_i)
+2iec^2A^i\partial_i
-V^2
-e^2c^2A^iA_i
+m^2c^4
\right)e^{-imc^2t}\varphi=
$$
$$=\left(
{\partial^2\over\partial t^2}
-c^2\nabla^2
+2iV{\partial\over\partial t}
+i{\partial V\over\partial t}
+iec^2(\partial^i A_i)
+2iec^2A^i\partial_i
-V^2
-e^2c^2A^iA_i
+m^2c^4
\right)e^{-imc^2t}\varphi=
$$
$$=e^{-imc^2t}\left(
(-imc^2+{\partial\over\partial t})^2
-c^2\nabla^2
+2iV(-imc^2+{\partial\over\partial t})
+i{\partial V\over\partial t}
+iec^2(\partial^i A_i)
+2iec^2A^i\partial_i
-V^2+ \right.
$$
$$\left.
-e^2c^2A^iA_i
+m^2c^4
\right)\varphi=
$$
$$=e^{-imc^2t}\left(
-2imc^2{\partial\over\partial t}+{\partial^2\over\partial t^2}
-c^2\nabla^2
+2Vmc^2
+2iV{\partial\over\partial t}
+i{\partial V\over\partial t}
+iec^2(\partial^i A_i)
+2iec^2A^i\partial_i
-V^2+ \right.
$$
$$\left.
-e^2c^2A^iA_i
\right)\varphi=
$$
$$=
-2mc^2 e^{-imc^2 t} \left(i{\partial\over\partial t}+{\nabla^2\over2m}-V
-{1\over2mc^2}{\partial^2\over\partial t^2}-{i\over2mc^2}{\partial
V\over\partial t}+{V^2\over2mc^2}-{iV\over mc^2}{\partial\over\partial
t}+\right.$$
$$\left.-{ie\over2m}\partial^i A_i-{ie\over
m}A^i\partial_i+{e^2\over2m}A^iA_i\right)\varphi
$$
Taking the limit $c\to\infty$ we again recover the Schr\"odinger equation:
$$i{\partial\over\partial t}\varphi=\left(-{\nabla^2\over2
m}+V
+{ie\over2m}\partial^i A_i
+{ie\over m}A^i\partial_i
-{e^2\over2m}A^iA_i
\right)\varphi\,,$$
we rewrite the right hand side a little bit:
$$i{\partial\over\partial t}\varphi=\left({1\over2 m}
(\partial^i\partial_i
+ie\partial^i A_i
+2ieA^i\partial_i
-e^2A^iA_i
)
+V
\right)\varphi\,,$$
$$i{\partial\over\partial t}\varphi=\left({1\over2 m}
(\partial^i+ieA^i)(\partial_i+ieA_i)
+V
\right)\varphi\,,$$
And we get the usual form of the Schr\"odinger equation for the vector
potential ${\bf A}=(A_1, A_2, A_3)$:
$$i{\partial\over\partial t}\varphi=\left(-{(\nabla+ie{\bf A})^2\over2 m}
+V \right)\varphi\,.$$

\section{Radial Schr\"odinger and Dirac Equations}

For general treatment, together with derivation of all the different forms of
radial Dirac equations used in the literature, see \cite{bachelor-thesis}. Here
we just summarize the results.

\subsection{Radial Schr\"odinger Equation}

We have a spherically symmetric potential energy
$$V({\bf x})=V(r)\,.$$
State with a given square of an angular momentum (eigenvalue
$l(l+1)$) and its $z$ component (eigenvalue $m$) is described by the wave
function
$$\psi_{nlm}({\bf x})=R_{nl}(r)\,Y_{lm}\left({\bf x}\over r\right)\,,\no{psi}$$
where $R_{nl}(r)$ obeys the equation \cite{formanek} (eq. 2.400)
$$R_{nl}''+{2\over r}R_{nl}'+{2M\over\hbar^2}(E-V)R_{nl}-
{l(l+1)\over r^2}R_{nl}=0\,.\no{radial}$$
This is called the radial Schr\"odinger equation which
we want to solve numerically.

\subsection{Numerical integration for a given $E$}

Equation \rno{radial} is the linear ordinary differential equation of the second
order, so the general solution is a linear combination of two independent
solutions. Normally, the $2$ constants are determined from initial and/or
boundary conditions. In our case, however, we don't have any other condition
besides being interested in solutions that we can integrate on the interval
$(0,\infty)$ (and which are normalizable), more exactly we want
$R\in L^2$ and $\int_0^\infty r^2R^2\,\d r=1$. 

It can be easily shown by a direct substitution, that there are only two
asymptotic behaviors near the origin: $r^l$ and $r^{-l-1}$. We are interested
in quadratic integrable solutions only, so we are left with $r^l$
and only one integration constant, which we calculate from a normalization.
This determines the solution uniquely.

All the integration algorithms needs to evaluate $R''$, which is a
problem at the origin, where all the terms in the equation are infinite,
although their sum is finite. We thus start to integrate the equation at some
small $r_0$ (for example $r_0=10^{-10}\rm\,a.u.$), where all the terms in the
equation are finite. If we find the initial conditions $R(r_0)$ and
$R'(r_0)$, the solution is then fully determined.

If $r_0$ is sufficiently small, we can set $R(r_0)=r_0^l$ and
$R'(r_0)=lr_0^{l-1}$. In the case $l=0$ we need to set $R(r_0)=1$ and
$R'(r_0)=-{1\over a}$, where $a$ is the Bohr radius, see the next
section for more details.

So when somebody gives us $l$ and $E$, we are now able to compute the solution
up to the the multiplicative constant that is later determined from a
normalization. As was already mentioned, we used the fourth-order Runge-Kutta
method that proved very suitable for this problem. 


\subsection{Asymptotic behavior}

The asymptotic behavior is important for the integration routine to find
the correct solution for a given $E$. 
It is well known, that the first
term of the Taylor series of the solution is $r^l$, independent
of the potential \cite{formanek} (eq. 2.408). This is enough
information to find the correct solution for $l>0$ because the only
thing we need to know is the value of the wave function and its derivative
near the origin, which is effectively $r_0^l$ and $lr_0^{l-1}$ for some small
$r_0$. The problem is with $l=0$, where the
derivative cannot be calculated just from $l$ and $r_0$. 

The asymptotic behavior for $l=0$ depends on the potential $V$, so we need to
take into account it's properties. We assume $V$ to be of a form:
$$V=-{Z\over r}+v_0 + v_1r + O(r^2)\,,$$
It can be shown, that the solution is then
$$R=a_0(1-{r\over a}+O(r^2))\,,$$
where $a={\hbar^2\over ZM}$ is the Bohr radius and $a_0$ is a normalization
constant. So the initial condition for the integration for $l=0$ is $R(r_0)=1$
and
$R'(r_0)=-{1\over a}$.

\subsection{Dirac Equation}

The Dirac equation for one particle is \cite{strange,zabloudil}:
$$H\psi=W\psi\,,\no{diraceq}$$
$$H=c\balpha\cdot{\bf p}+\beta mc^2+V(r)\hbox{\dsrom1}\,,$$
where $\psi$ is a four component vector:
$$\psi=\left(\matrix{\psi_1\cr\psi_2\cr\psi_3\cr\psi_4\cr}\right)
=\col{\psi_A}{\psi_B}\,,\qquad
\psi_A=\col{\psi_1}{\psi_2}\!,\,\psi_B=\col{\psi_3}{\psi_4}$$
and $\balpha$, $\beta$ are $4\times4$ matrices:
$$\balpha=\mat{0}{\bsigma}{\bsigma}{0}\,,$$
$$\beta=\mat{\hbox{\dsrom1}}{0}{0}{-\hbox{\dsrom1}}\,,$$
where the Pauli matrices $\bsigma=(\sigma_x,\sigma_y,\sigma_z)$ and
$\hbox{\dsrom1}$ form a basis of all $2\times2$ Hermitian matrices.
To derive a continuity equation, we multiply \rno{diraceq} by $\psi^*$
and subtract the conjugate transpose of \rno{diraceq} multiplied by $\psi$:
$$\p{}{t}(\psi^*\psi)=-\nabla\cdot(c\psi^*\balpha\psi)\,,$$
so we identify the probability and current densities as
$$\rho=\psi^*\psi=\psi_1^*\psi_1+\psi_2^*\psi_2+\psi_3^*\psi_3+\psi_4^*\psi_4\,,
\qquad {\bf j}=c\psi^*\balpha\psi\,.$$
The normalization of a four-component wave function is then
$$
\int \rho \,\d^3x=
\int \psi^*\psi \,\d^3x=
\int \psi_1^*\psi_1+\psi_2^*\psi_2+\psi_3^*\psi_3+\psi_4^*\psi_4 \,\d^3x=
1\,.\no{norm}$$
The probability density $\rho(x,y,z)$ is the physical quantity we are
interested in, and all the four-component wavefunctions and other formalism is
just a way of calculating it. This $\rho$ is also the thing we should compare
with the Schr\"odinger equation. 

\subsection{Radial Dirac equation}

We and search for a basis in the form of spin angular functions:
$$\psi_A=g\chi^{j_3}_{\kappa}\,,\no{psia}$$
$$\psi_B=if\chi^{j_3}_{-\kappa}\,.\no{psib}$$
Substituting all of these into \rno{diraceq} and some more well-known
manipulations one gets:
$$\hbar c
\col{-\p{f}{r}+{\kappa-1\over r}f}
{\p{g}{r}+{\kappa+1\over r}g}=
\col{(W-V-mc^2)g}{(W-V+mc^2)f}\,.\no{radialdirac}$$
This is the radial Dirac equation. As we shall see in the next section, the
equation for $g$ is (with the exception of a few relativistic corrections)
identical to the radial Schr\"odinger equation. And $f$ vanishes
in the limit $c\to\infty$. For this reason $f$ is called the small 
(fein, minor) component and $g$ the large (gro\ss, major) component. 

The probability density is
$$\rho=\psi^*\psi=\psi^*_A\psi_A+\psi^*_B\psi_B=
f^2\chi^{j_3*}_{-\kappa}\chi^{j_3}_{-\kappa}+
g^2\chi^{j_3*}_{\kappa}\chi^{j_3}_{\kappa}\,,
$$
so from the normalization condition \rno{norm} we get
$$
\int \rho \,\d^3x=
\int f^2\chi^{j_3*}_{-\kappa}\chi^{j_3}_{-\kappa}+
g^2\chi^{j_3*}_{\kappa}\chi^{j_3}_{\kappa} \,\d^3x=
\int (f^2\chi^{j_3*}_{-\kappa}\chi^{j_3}_{-\kappa}+
g^2\chi^{j_3*}_{\kappa}\chi^{j_3}_{\kappa})\,r^2\,\d r\d\Omega=
$$
$$
=\int_0^\infty f^2r^2\,\d r\int\chi^{j_3*}_{-\kappa}\chi^{j_3}_{-\kappa}
\,\d\Omega+
\int_0^\infty g^2r^2\,\d r\int\chi^{j_3*}_{\kappa}\chi^{j_3}_{\kappa}
\,\d\Omega=
\int_0^\infty r^2(f^2+g^2)\,\d r=1\,,
$$
where we used the normalization of spin-angular functions.
Also it can be seen, that the radial probability density
is 
$$\rho(r)=r^2(f^2+g^2)\no{radialrho}$$
(i.e., the probability to find the electron
between $r_1$ and $r_2$ is $\int_{r_1}^{r_2}r^2(f^2+g^2)\,\d r$). 
The result of integrating the radial
Dirac equation are the two functions $f$ and $g$, but the
physically relevant quantity is the radial probability density \rno{radialrho}.
In the 
nonrelativistic case, the density is given by
$$\rho(r)=r^2R^2\,,$$
so the correspondence between the Schr\"odinger and Dirac equation is 
$R^2=f^2+g^2$. 

For numerical stability and robustness, we are not solving the equations in the
form \rno{radialdirac}, but a sligthly rearranged ones. Let's use Hartree
atomic units ($m=\hbar=1$) and define $E=W-mc^2=W-c^2$, so that $E$ doesn't
contain the electron rest mass energy. 
Let's make the substitution \cite{donald:apw}
$$\eqalign{
P_\kappa&=rg_\kappa\,,\cr
Q_\kappa&=rf_\kappa\cr
}$$
and plug all of this into \rno{radialdirac}. After a little manipulation we
get:
$$\eqalign{
{\d P_\kappa\over\d r}&=-{\kappa\over r}P_\kappa+\left[{E-V\over c}+2c\right]Q_k\,,\cr
{\d Q_\kappa\over\d r}&={\kappa\over r}Q_\kappa-{1\over c}(E-V)P_k\,,
\cr\nno{radialdirac2}
}$$
which can be found in \cite{zabloudil} (eq. 8.12 and 8.13), where they have one
$c$
hidden in $Q_\kappa=crf_\kappa$ and use Rydberg atomic units, so they have $1$ instead of $2$ in the square bracket.
It can be found in \cite{bachelet} as well, they use Hartree atomic units, but
have a different notation $G_\kappa\equiv P_\kappa$ and $F_\kappa\equiv
Q_\kappa$, also they made a substitution $c={1\over\alpha}$.

\subsection{Asymptotic behavior}

We calculate the functions $f_\kappa$ and $g_\kappa$ in a similar
way as we calculated $R$ for the Schr\"odinger equation, thus we 
need the asymptotic behavior at the origin. The potential
can always be treated as $V=1/r+\cdots$ and in this case 
it can be shown \cite{zabloudil}, that the asymptotic is 
$$P_\kappa = r g_\kappa=r^{\beta}\,,$$
$$Q_\kappa = r f_\kappa=r^{\beta-1}{\beta+\kappa\over{E-V\over c}+2c}\,,$$
where 
$$\beta=\sqrt{\kappa^2-\left(Z\over c\right)^2}\,,\no{diracasymptotic}$$
or, if we write it explicitly, for $j=l+\half$
$$\beta^+=\sqrt{(-l-1)^2-\left(Z\over c\right)^2}$$
and $j=l-\half$
$$\beta^-=\sqrt{l^2-\left(Z\over c\right)^2}\,.$$
In the semirelativistic case (which is an approximation --- we neglect
the spin-orbit coupling term) we choose
$$\beta=\sqrt{\half(|\beta^+|^2+|\beta^-|^2)}=
\sqrt{l^2+l+\half-\left(Z\over c\right)^2}\,.$$
It should be noted that in the literature we can find other types of 
aymptotic behaviour for the semirelativistic case, its just a question of the
used approximation. One can hardly say that some of them are correct and
another is not since the semirelativistic (sometimes denoted as
scalar-relativstic) approximation itself is not correct, it's just an
approximation.

It follows from \rno{diracasymptotic} that for $j=l+\half$ the radial Dirac
equation completely becomes the radial Schr\"odinger equation in the limit
$c\to\infty$ (and gives exactly the same solutions):
$$P_\kappa = r g_\kappa \to r^{l+1}\,,$$
$$Q_\kappa = r f_\kappa \to 0\,.$$
For $j=l-\half$ however, 
we get a wrong asymptotic: we get a radial Schr\"odinger equation for $l$, but
the asymptotic for $l-1$.


\subsection{Eigenproblem}

In the previous sections, we learned how to calculate the solution of both
the radial Schr\"odinger and Dirac equations for a given $E$.  For most of the
energies, however, the solution for $r\to\infty$ exponentially diverges to
$\pm\infty$. Only for the energies equal to eigenvalues, the solution tends
exponentially to zero for $r\to\infty$. The spectrum for bounded states is
discrete, so we label the energies by $n$, starting from $1$. 

We want to find the eigenvalue and eigenfunction for a given $n$ and $l$
(and a spin in the relativistic case).
The algorithm is the same for both nonrelativistic and relativistic case and
is based on two facts, first that the number of nodes (ie. the number of
intersections with the $x$ axis, not counting the one at the origin and in the
infinity) of $R_{nl}$ and $g_\kappa$ is $n-l-1$ and second that the solution
must tend to zero at infinity.

We calculate the solution for some (random) energy $E_0$, using the
procedure described above. Then we count the number of nodes 
(for diverging solutions, we don't count the last one) and check, if the
solution is approaching the zero from top or bottom in the infinity. From the
number of nodes and the direction it is approaching the zero it can be
determined
whether the energy $E_0$ is below or above the eigenvalue $E$ belonging to a given
$n$ and $l$. The rest is simple, we find two energies, one below $E$, one above
$E$ and by bisecting the interval we calculate $E$ with any precision we want.

There are a few technical numerical problems that are unimportant from the
theoretical point of view, but that need to be solved if one attempts to
actually implement this algorithm. One of them is that when the algorithm
(described in the previous paragraph) finishes, because the energy interval is
sufficiently small, it doesn't mean
the solution is near zero for the biggest $r$ of our grid.
Remember, the solution goes exponentially to $\pm\infty$ for every $E$ except
the eigenvalues and because we never find the exact eigenvalue, the solution
will (at some point) diverge from zero. 

Possible solution that we have employed is as follows: 
when the algorithm finishes we find the last minimum (which is always near
zero) and trim the solution behind it (set it to zero). 

The second rather technical problem is how to choose the initial interval of
energies so that the eigenvalue lies inside the interval. We use some default
values that work for atomic calculations, while allowing the user to override
it if needed. 
