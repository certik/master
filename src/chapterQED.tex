\chapter{Derivation from QED}

\section{Introduction}

The correct theory for electrons is the quantum electrodynamics (QED). Thus all
the equations which we are going to use, follow from the QED and this chapter
is devoted for their derivation. The standard nonrelativistic quantum
mechanics (QM)
contains a lot of ad-hoc assumptions (Schr\"odinger equation, electromagnetic
coupling, spin, etc.), that cannot be satisfactorily explained. It is true that
the whole nonrelativistic QM can be derived from a small set of
assumptions (mostly commutation relations and some particular form of QM
operators like spin), but it is not at all clear why these assumptions look the
way they look. The QM is a set of rules which are known to work quite
well, but obviously, we are not interested in postulating correct forms of
operators or commutation relations, we want a better explanation, if there is
one. And fortunately, there is.

The correct theory so far is the Standard Model and in our case of electron
structure calculations, all phenomena are explained by the QED. There are still
some problems in the Standard Model, but for much higher energies than
we deal with in the electron structure. Thus what we do is that we take
the most complete theory so far (QED) and derive everything from it. Of course,
because we are dealing with low energies and to ease calculations, we make a
lot of approximations on our way, but this is the only correct way to
proceed, because it will be clear that we make each approximation because of
this and that. If it turns out our results are not in agreement with experiment
(for example the Schr\"odinger equation is imprecise), we need to neglect
less things. 

As you will see, this chapter is not long, so there really is no excuse of
doing things differently (maybe in 1940s, but certainly not now when we have
the Standard Model). 



\def\L{L}

\section{QED}

The QED Lagrangian density is
$$\L=\bar\psi(ic\gamma^\mu D_\mu-mc^2)\psi-{1\over4}F_{\mu\nu}F^{\mu\nu}$$
where
$$D_\mu=\partial_\mu+ieA_\mu$$
is the gauge covariant derivative and ($e$ is the elementary charge, which is
$1$ in atomic units)
$$F_{\mu\nu}=\partial_\mu A_\nu-\partial_\nu A_\mu$$
is the electromagnetic field tensor. It's astonishing, that this simple
Lagrangian can account for all phenomena from macroscopic scales down to
something like $10^{-13}\rm\,cm$. So of course Feynman, Schwinger and Tomonaga
received the 1965 Nobel Prize in Physics for such a fantastic achievement.

Plugging this Lagrangian into the Euler-Lagrange equation of motion for a
field, we get:
$$(ic\gamma^\mu D_\mu-mc^2)\psi=0$$
$$\partial_\nu F^{\nu\mu}=-ec\bar\psi\gamma^\mu\psi$$
The first equation is the Dirac equation in the electromagnetic field and
the second equation is a set of Maxwell equations ($\partial_\nu
F^{\nu\mu}=-ej^\mu$) with a source $j^\mu=c\bar\psi\gamma^\mu\psi$, which is a
4-current comming from the Dirac equation.

The fields $\psi$ and $A^\mu$ are quantized. The first approximation is that we
take $\psi$ as a wavefunction, that is, it is a classical 4-component field. It
can be shown that this corresponds to taking three orders in the perturbation
theory.

$$A_\mu=\left({V\over ec},A_1,A_2,A_3\right)$$
$$0=\gamma^0(ic\gamma^\mu D_\mu-mc^2)\psi=
\gamma^0(ic\gamma^0(\partial_0+i{V\over c})+ic\gamma^i
(\partial_i+ieA_i)-mc^2)\psi=$$
$$=
(ic\partial_0+ic\gamma^0\gamma^i\partial_i-\gamma^0mc^2-V
-ce\gamma^0\gamma^iA_i)\psi
$$
$\beta=\gamma^0$, $\alpha^i=\gamma^0\gamma^i$, $p_j=-i\partial_j$,
$\partial_0={1\over c}{\partial\over\partial t}$
$$ (ic\partial_0-c\alpha^i p_i-\beta mc^2-V-ce\alpha^iA_i)\psi=0 $$
$$ i{\partial\psi\over\partial t}=(c\alpha^i(p_i+eA_i)+\beta mc^2+V)\psi $$

$$j^\mu=c\bar\psi\gamma^\mu\psi$$

$\psi=e^{-imt}\varphi$, ${\partial\over\partial t}e^{-imt}(\cdots)=
e^{-imt}(-im+{\partial\over\partial t})(\cdots)$
$$j^0=c\psi^\dagger\psi=c\varphi^\dagger\varphi$$
$$j^i=c\psi^\dagger\alpha^i\psi=c\varphi^\dagger\alpha^i\varphi$$

$$(-ic\gamma^\mu D_\mu-mc^2)(ic\gamma^\nu D_\nu-mc^2)\psi=
(c^2D^\mu D_\mu+m^2c^4)\psi=0$$
$$D^\mu D_\mu=\partial^\mu\partial_\mu+ie(\partial^\mu
A_\mu+2A^\mu\partial_\mu) -e^2A^\mu A_\mu$$
$$D^\mu D_\mu=\partial^\mu\partial_\mu
+ie\left({1\over c^2}{\partial V\over\partial t}+
2{V\over c^2}{\partial\over\partial t}\right) -{e^2\over c^2}V^2$$

$$L=c^2\partial^\mu\psi^*\partial_\mu\psi-m^2c^4\psi^*\psi=
{\partial\over\partial t}\psi^*{\partial\over\partial t}\psi
-c^2\partial^i\psi^*\partial_i\psi-m^2c^4\psi^*\psi=$$
$$=(imc^2+{\partial\over\partial t})\varphi^*
(-imc^2+{\partial\over\partial t})\varphi
-c^2\partial^i\varphi^*\partial_i\varphi-m^2c^4\varphi^*\varphi=$$
$$=2mc^2\left[{1\over2}i(\varphi^*{\partial\varphi\over\partial t}-
\varphi{\partial\varphi^*\over\partial t})-
{1\over2m}\partial^i\varphi^*\partial_i\varphi
+{1\over2mc^2}{\partial\varphi^*\over\partial t}
{\partial\varphi\over\partial t}\right]$$
Neglecting the last term we get
$$L={1\over2}i(\varphi^*{\partial\varphi\over\partial t}-
\varphi{\partial\varphi^*\over\partial t})-
{1\over2m}\partial^i\varphi^*\partial_i\varphi
$$
After integration by parts we arrive at
$$L=i\varphi^*{\partial\varphi\over\partial t}
-{1\over 2m}\partial^i\varphi^*\partial_i
\varphi$$
The nonrelativistic limit can also be applied directly to the Klein-Gordon equation:
$$0=(c^2D^\mu D_\mu+m^2c^4)\psi=
(c^2\partial^0\partial_0-c^2\partial^i\partial_i+iec^2(\partial^\mu
A_\mu+2A^\mu\partial_\mu) -e^2c^2A^\mu A_\mu+m^2c^4)e^{-imc^2t}\varphi=$$
$$=e^{-imc^2t}((-imc^2+{\partial\over\partial t})^2-c^2\nabla^2
+iec^2\partial^\mu A_\mu+2emcA^0+2iec A^0{\partial\over\partial
t}+2iec^2A^i\partial_i-e^2c^2 A^\mu A_\mu
+m^2c^4)\varphi=$$
$$=
-2mc^2e^{-imc^2t}\left(i{\partial\over\partial
t}-{1\over2mc^2}{\partial^2\over\partial t^2}+{\nabla^2\over2m}
-eA^0+{e^2\over2mc^2}A^\mu A_\mu
-{ie\over2m}(\partial^\mu A\mu+2A^\mu\partial_\mu)
\right)\varphi$$
Neglecting the second time derivative, we again get the Schr\"odinger equation:
$$i{\partial\over\partial t}\varphi=-{\nabla^2\over2 m}\varphi$$

Zkontrolovat predposledni radek, jsou tam blbe ccka.
