In this section we show results of our program. The program is opensource (BSD
licensed) and available at \url{http://sfepy.org}, use the release 00.50.00 to
get the exact results as below.

\section{Solution of the Schr\"odinger Equation}

The first thing we need to do is to create a mesh.

\subsection{2D: Mesh}

We create a mesh:

\begin{lstlisting}
$ ./schroedinger.py --mesh --2d
Dimension: 2
Info    : 'gmsh -2 tmp/mesh.geo -format mesh'
   started on Sat Jul 19 02:04:41 2008
Info    : Reading 'tmp/mesh.geo'
Info    : Read 'tmp/mesh.geo'
Info    : Meshing 1D...
Info    : Meshing curve 1 (Line)
Info    : Meshing curve 2 (Line)
Info    : Meshing curve 3 (Line)
Info    : Meshing curve 4 (Line)
Info    : Mesh 1D complete (0.028002 s)
Info    : Mesh
Info    : Meshing 2D...
Info    : Meshing surface 6 (Plane, MeshAdapt+Delaunay)
Info    : Mesh 2D complete (3.65223 s)
Info    : Mesh
Info    : 8100 vertices 16198 elements
Info    : Writing 'tmp/mesh.mesh'
Info    : Wrote 'tmp/mesh.mesh'
Mesh written to tmp/mesh.vtk
\end{lstlisting}

As you can see, we call gmsh\cite{gmsh} in the background to create a
triangular mesh in 2D.
See the fig. \ref{fig:mesh2d} and \ref{fig:mesh2d2} for a visualization of the
mesh in the paraview\cite{paraview} program.

\def\fig#1#2#3{%
\begin{figure}[htp]
\centering
\includegraphics[width=12cm]{#1}
\caption{#3}\label{#2}
\end{figure}
}

\fig{images/mesh1-r.pdf}{fig:mesh2d}{2D mesh, the red square is zoomed in the fig.  \ref{fig:mesh2d2}}

\fig{images/mesh4.png}{fig:mesh2d2}{Zoomed in 2D mesh}

\subsection{2D: Potential Well}

This is also sometimes called particle in the box.

Our program constructs the element matrices as described before and then calls
PySparse\cite{geus} to solve the generalized eigenproblem.

\begin{lstlisting}
$ ./schroedinger.py --well
Dimension: 2
sfepy: left over: ['tau', 'n_eigs', 'mesh',
    'quadr', 'dim', 'base_approx']
sfepy: reading mesh (tmp/mesh.vtk)...
sfepy: ...done in 0.37 s
sfepy: setting up domain edges...
sfepy: ...done in 0.10 s
sfepy: creating regions...
sfepy:     leaf Omega region_1000
sfepy:     leaf Surface region_2
sfepy: ...done in 0.11 s
sfepy: equation "rhs":
sfepy: dw_mass_scalar.i1.Omega( v, Psi )
sfepy: equation "lhs":
sfepy:   dw_laplace.i1.Omega( m.val, v, Psi )
   + dw_mass_scalar_variable.i1.Omega( matV.V, v, Psi )
sfepy: describing geometries...
sfepy: ...done in 0.02 s
sfepy: setting up dof connectivities...
sfepy: ...done in 0.00 s
sfepy: matrix shape: (8072, 8072)
sfepy: assembling matrix graph...
sfepy: ...done in 0.02 s
sfepy: matrix structural nonzeros: 56442 (8.66e-04% fill)
sfepy: updating materials...
sfepy:     m
sfepy:     matV
sfepy: ...done in 0.00 s
sfepy: assembling lhs...
sfepy:   setting up dof connectivities...
sfepy:   ...done in 0.00 s
sfepy: ...done in 0.04 s
sfepy: assembling rhs...
sfepy:   setting up dof connectivities...
sfepy:   ...done in 0.00 s
sfepy: ...done in 0.02 s
computing resonance frequencies...
sfepy: loading...
sfepy: ...done
sfepy: solving...
sfepy:   number of converged eigenvalues: 10
sfepy: ...done in 3.10 s
sfepy: reading mesh (tmp/mesh.vtk)...
sfepy: ...done in 0.42 s
a=100.000000
Energies:
n      exact         FEM      error
0:  0.00098696   0.00100813   2.14%
1:  0.00246740   0.00255738   3.65%
2:  0.00246740   0.00256454   3.94%
3:  0.00394784   0.00421025   6.65%
4:  0.00493480   0.00524481   6.28%
5:  0.00493480   0.00525660   6.52%
6:  0.00641524   0.00705958  10.04%
7:  0.00641524   0.00706794  10.17%
8:  0.00838916   0.00916761   9.28%
9:  0.00838916   0.00920827   9.76%
Solution saved to mesh.vtk
\end{lstlisting}

All the eigenvalues are visualized in fig. \ref{fig:well2d-1} -
\ref{fig:well2d-9}.

\fig{images/well1.png}{fig:well2d-1}{2D well: eigenvalue 1}

\fig{images/well2.png}{fig:well2d-2}{2D well: eigenvalue 2}

\fig{images/well3.png}{fig:well2d-3}{2D well: eigenvalue 3}

\fig{images/well4.png}{fig:well2d-4}{2D well: eigenvalue 4}

\fig{images/well5.png}{fig:well2d-5}{2D well: eigenvalue 5}

\fig{images/well6.png}{fig:well2d-6}{2D well: eigenvalue 6}

\fig{images/well7.png}{fig:well2d-7}{2D well: eigenvalue 7}

\fig{images/well8.png}{fig:well2d-8}{2D well: eigenvalue 8}

\fig{images/well9.png}{fig:well2d-9}{2D well: eigenvalue 9}

To get a better insight, we plotted the 6th eigenvalue as a 2D surface in 3D,
see the fig \ref{fig:well2d-6warp}.

\fig{images/well6warp.png}{fig:well2d-6warp}{2D well: eigenvalue 6, plotted as
a surface}

\subsection{2D: Linear Harmonic Oscillator}

\begin{lstlisting}
$ ./schroedinger.py --oscillator
Dimension: 2
[...]
Energies:
n      exact         FEM      error
0:  1.00000000   1.00081703   0.08%
1:  2.00000000   2.00158339   0.08%
2:  2.00000000   2.00175346   0.09%
3:  3.00000000   3.00302269   0.10%
4:  3.00000000   3.00334897   0.11%
5:  3.00000000   3.00346025   0.12%
6:  4.00000000   4.00498898   0.12%
7:  4.00000000   4.00571512   0.14%
8:  4.00000000   4.00606336   0.15%
9:  4.00000000   4.00631703   0.16%
10:  5.00000000   5.00743227   0.15%
11:  5.00000000   5.00868813   0.17%
12:  5.00000000   5.00947456   0.19%
13:  5.00000000   5.01031228   0.21%
14:  5.00000000   5.01042976   0.21%
15:  6.00000000   6.01161255   0.19%
Solution saved to mesh.vtk
\end{lstlisting}

All the eigenvalues are visualized in fig. \ref{fig:osc2d-1} -
\ref{fig:osc2d-14}.

\fig{images/osc1.png}{fig:osc2d-1}{2D oscillator: eigenvalue 1}

\fig{images/osc2.png}{fig:osc2d-2}{2D oscillator: eigenvalue 2}

\fig{images/osc3.png}{fig:osc2d-3}{2D oscillator: eigenvalue 3}

\fig{images/osc4.png}{fig:osc2d-4}{2D oscillator: eigenvalue 4}

\fig{images/osc5.png}{fig:osc2d-5}{2D oscillator: eigenvalue 5}

\fig{images/osc6.png}{fig:osc2d-6}{2D oscillator: eigenvalue 6}

\fig{images/osc7.png}{fig:osc2d-7}{2D oscillator: eigenvalue 7}

\fig{images/osc8.png}{fig:osc2d-8}{2D oscillator: eigenvalue 8}

\fig{images/osc9.png}{fig:osc2d-9}{2D oscillator: eigenvalue 9}

\fig{images/osc10.png}{fig:osc2d-10}{2D oscillator: eigenvalue 10}

\fig{images/osc11.png}{fig:osc2d-11}{2D oscillator: eigenvalue 11}

\fig{images/osc12.png}{fig:osc2d-12}{2D oscillator: eigenvalue 12}

\fig{images/osc13.png}{fig:osc2d-13}{2D oscillator: eigenvalue 13}

\fig{images/osc14.png}{fig:osc2d-14}{2D oscillator: eigenvalue 14}

To get a better insight, we plotted the 8th and 10th eigenvalue as a 2D surface
in 3D, see the figs \ref{fig:osc2d-8warp} and \ref{fig:osc2d-10warp}.

\fig{images/osc8warp.png}{fig:osc2d-8warp}{2D oscillator: eigenvalue 8, plotted as a surface}

\fig{images/osc10warp.png}{fig:osc2d-10warp}{2D oscillator: eigenvalue 10, plotted as a surface}

\subsection{2D: Hydrogen Atom}

\begin{lstlisting}
$ ./schroedinger.py --hydrogen
Dimension: 2
[...]
Energies:
n      exact         FEM      error
0:  -0.50000000   -0.48444312   3.11%
1:  -0.05555556   -0.05546226   0.17%
2:  -0.05555556   -0.05490833   1.17%
3:  -0.02000000   -0.01987759   0.61%
4:  -0.02000000   -0.01986552   0.67%
5:  -0.02000000   -0.01978261   1.09%
6:  -0.02000000   -0.01974034   1.30%
7:  -0.02000000   -0.01973349   1.33%
8:  -0.01020408   -0.00961296   5.79%
9:  -0.01020408   -0.00959015   6.02%
Solution saved to mesh.vtk
\end{lstlisting}

\subsection{2D: Boron Atom}

\begin{lstlisting}
$ ./schroedinger.py --boron
[...]
Energies:
n      exact         FEM      error
0:  -12.50000000   -12.65582408   1.25%
1:  -1.38888889   -1.39403240   0.37%
2:  -1.38888889   -1.38823512   0.05%
3:  -1.38888889   -1.38821053   0.05%
4:  -0.50000000   -0.50042416   0.08%
5:  -0.50000000   -0.49898923   0.20%
6:  -0.50000000   -0.49895079   0.21%
7:  -0.50000000   -0.49809201   0.38%
8:  -0.50000000   -0.49804906   0.39%
9:  -0.25510204   -0.25394635   0.45%
10:  -0.25510204   -0.25329618   0.71%
11:  -0.25510204   -0.25324683   0.73%
12:  -0.25510204   -0.25249085   1.02%
13:  -0.25510204   -0.25242286   1.05%
Solution saved to mesh.vtk
\end{lstlisting}

\subsection{3D: Mesh}

In 3D, we generate tetrahedra:

\begin{lstlisting}
$ ./schroedinger.py --mesh
Dimension: 3
[...]
Mesh written to tmp/mesh.vtk
\end{lstlisting}

Look at the fig. \ref{fig:mesh3d} to see how the mesh looks like.

\fig{images/mesh3d.png}{fig:mesh3d}{3D mesh}

\subsection{3D: Potential Well}

$$V(x)=\begin{cases}0,&\text{inside the box}\quad a\times a\times a \\
\infty, & \text{outside}\end{cases}$$
Analytic solution:
$$E_{n_1n_2n_3}={\pi^2\over2a^2}\left(n_1^2 + n_2^2 + n_3^2\right)$$
where $n_i = 1, 2, 3, \dots$ are independent quantum numbers.
We chose $a=1$, i.e.: $E_{111}=14.804$, $E_{211}=E_{121}=E_{112}=29.608$,
$E_{122}=E_{212}=E_{221}=44.413$,
$E_{311}=E_{131}=E_{113}=54.282$
$E_{222}=59.217$, $E_{123}=E_{\hbox{perm.}}=69.087$.

\begin{lstlisting}
$ ./schroedinger.py --well
Dimension: 3
[...]
Energies:
n      exact         FEM      error
0:  0.14804407   0.14922535   0.80%
1:  0.29608813   0.30079010   1.59%
2:  0.29608813   0.30082698   1.60%
3:  0.29608813   0.30084093   1.61%
4:  0.44413220   0.45473187   2.39%
5:  0.44413220   0.45482735   2.41%
6:  0.44413220   0.45489304   2.42%
7:  0.54282824   0.55869467   2.92%
8:  0.54282824   0.55871268   2.93%
9:  0.54282824   0.55889291   2.96%
10:  0.59217626   0.61113461   3.20%
11:  0.69087231   0.71656267   3.72%
12:  0.69087231   0.71661781   3.73%
13:  0.69087231   0.71678728   3.75%
14:  0.69087231   0.71695827   3.78%
15:  0.69087231   0.71706618   3.79%
16:  0.69087231   0.71729282   3.82%
17:  0.83891637   0.87708704   4.55%
18:  0.83891637   0.87715932   4.56%
19:  0.83891637   0.87797932   4.66%
Solution saved to mesh.vtk
\end{lstlisting}

As you can see above, we got ($a=1$, 24702 nodes):

\begin{tabular}{ c | c c c c c c }
E      & 1 & 2-4 & 5-7 & 8-10 & 11 & 12- \\
\hline
theory & 14.804 & 29.608 & 44.413 & 54.282 & 59.217 & 69.087 \\
FEM    & 14.861 & 29.833 & 44.919 & 55.035 & 60.123 & 70.305 \\
       &        & 29.834 & 44.920 & 55.042 &        & 70.310 \\
       &        & 29.836 & 44.925 & 55.047 &        & $\cdots$ \\
\end{tabular}

So we got the correct energies and correct degeneracies. See the figs
\ref{fig:well3d1} -- \ref{fig:well3d10c} for examples of eigenvalues.

\fig{images/well3d1.png}{fig:well3d1}{3D potential well: eigenvalue 1, cut
plane}

\fig{images/well3d1c.png}{fig:well3d1c}{3D potential well: eigenvalue 1, cut
plane and contours}

\fig{images/well3d10c.png}{fig:well3d10c}{3D potential well: eigenvalue 10, cut
plane and contours}

\subsection{3D: Linear Harmonic Oscillator}

$$V(r)=\begin{cases}\half\omega^2r^2,&\text{inside the box}\quad a\times a\times a \\
\infty, & \text{outside}\end{cases}$$
Analytic solution in the limit $a\to\infty$:
$$E_{nl}=\left(2n+l+{3\over2}\right)\omega$$
where $n, l = 0, 1, 2, \dots$.
Degeneracy is $2l+1$, so:
$E_{00} = {3\over2}$,
triple $E_{01} = {5\over2}$,
$E_{10} = {7\over2}$,
quintuple $E_{02} = {7\over2}$,
triple $E_{11} = {9\over2}$,
quintuple $E_{12} = {11\over2}$:

\begin{lstlisting}
$ ./schroedinger.py --oscillator
Dimension: 3
[...]
Energies:
n      exact         FEM      error
0:  1.50000000   1.60246586   6.83%
1:  2.50000000   2.66350384   6.54%
2:  2.50000000   2.66592179   6.64%
3:  2.50000000   2.66745129   6.70%
4:  3.50000000   3.73764474   6.79%
5:  3.50000000   3.74251455   6.93%
6:  3.50000000   3.74545592   7.01%
7:  3.50000000   3.74824003   7.09%
8:  3.50000000   3.75739243   7.35%
9:  3.50000000   3.78188467   8.05%
10:  4.50000000   4.84098311   7.58%
11:  4.50000000   4.84411120   7.65%
12:  4.50000000   4.84695596   7.71%
13:  4.50000000   4.85068992   7.79%
14:  4.50000000   4.85439463   7.88%
15:  4.50000000   4.86152828   8.03%
16:  4.50000000   4.86513080   8.11%
17:  4.50000000   4.91355134   9.19%
18:  4.50000000   4.91666221   9.26%
19:  4.50000000   4.92487984   9.44%
Solution saved to mesh.vtk
\end{lstlisting}

Numerical solution ($a=15$, $\omega=1$, 290620 nodes):

\begin{tabular}{ c | c c c c }
E      & 1 & 2-4 & 5-10 & 11- \\
\hline
theory & 1.5 & 2.5 & 3.5 & 4.5 \\
FEM    & 1.522 & 2.535 & 3.554 & 4.578 \\
       &        & 2.536 & 3.555 & 4.579 \\
       &        & 2.536 & 3.555 & 4.579 \\
       &        &       & 3.555 &  $\cdots$ \\
       &        &       & 3.556 &   \\
       &        &       & 3.556 &   \\
\end{tabular}

For better imagination, we plotted the eigenvalue 5 from several sides and then
eigenvalue 10, see the figs \ref{fig:osc3d5} -- \ref{fig:osc3d10}.

\fig{images/osc3d5.png}{fig:osc3d5}{3D oscillator: eigenvalue 5, cut
plane}

\fig{images/osc3d5c.png}{fig:osc3d5c}{3D oscillator: eigenvalue 5, cut
plane and countour}

\fig{images/osc3d5c2.png}{fig:osc3d5c2}{3D oscillator: eigenvalue 5,
countour}

\fig{images/osc3d10.png}{fig:osc3d10}{3D oscillator: eigenvalue 10, cut
plane and contour}

\subsection{3D: Hydrogen Atom}

$$V(r)=\begin{cases}-{1\over r},&\text{inside the box}\quad a\times a\times a \\
\infty, & \text{outside}\end{cases}$$
Analytic solution in the limit $a\to\infty$:
$$E_n=-{1\over2n^2}$$
where $n=1, 2, 3, \dots$.
Degeneracy is $n^2$, so:
$E_1 = -{1\over2}=-0.5$,
$E_2 = -{1\over8}=-0.125$,
$E_3 = -{1\over18}=-0.055$,
$E_4 = -{1\over32}=-0.031$.

\begin{lstlisting}
$ ./schroedinger.py --hydrogen
Dimension: 3
[...]
Energies:
n      exact         FEM
0:  -0.50000000   -0.13468961
1:  -0.12500000   0.13909268
2:  -0.12500000   0.13934116
3:  -0.12500000   0.13939501
4:  -0.12500000   0.26835117
Solution saved to mesh.vtk
\end{lstlisting}

As you can see above, our mesh is not sufficient to get precise results, so we
used a refined mesh with $a=15$ and 160000 nodes and we got more precise
results:

\begin{tabular}{ c | c c c c }
E      & 1 & 2-5 & 6-14 & 15- \\
\hline
theory & -0.5 & -0.125 & -0.055 & -0.031 \\
FEM    & -0.481 & -0.118 & -0.006 & $\cdots$ \\
\end{tabular}

\subsection{2D: nonsymmetric potential I}

In this example we use a potential from two nuclei positioned at $(-5, 0)$ and
$(5, 0)$. This is a nonsymmetric problem, thus one cannot use the usual
way to reduce the Schr\"odinger equation to radial and angular
parts. A general partial differential equations solver (in our case FEM) has to
be used. See the figs \ref{fig:nonsym2-first} --
\ref{fig:nonsym2-last}.

\fig{images/nonsym20.png}{fig:nonsym2-first}{2D nonsymmetric potential I:
eigenvalue 0}

\fig{images/nonsym21.png}{}{2D nonsymmetric potential I: eigenvalue 1}

\fig{images/nonsym22.png}{}{2D nonsymmetric potential I: eigenvalue 2}

\fig{images/nonsym23.png}{}{2D nonsymmetric potential I: eigenvalue 3}

\fig{images/nonsym24.png}{}{2D nonsymmetric potential I: eigenvalue 4}
% this is needed, otherwise latex cries on the fig below.
\clearpage

\fig{images/nonsym25.png}{}{2D nonsymmetric potential I: eigenvalue 5}

\fig{images/nonsym26.png}{}{2D nonsymmetric potential I: eigenvalue 6}

\fig{images/nonsym27.png}{}{2D nonsymmetric potential I: eigenvalue 7}

\fig{images/nonsym28.png}{}{2D nonsymmetric potential I: eigenvalue 8}

\fig{images/nonsym29.png}{fig:nonsym2-last}{2D nonsymmetric potential I: eigenvalue 9}

\subsection{2D: nonsymmetric potential II}

In this example we use a potential from three nuclei positioned at $(-5, 0)$,
$(5, 0)$ and $(0, 5)$. See the figs \ref{fig:nonsym-first} --
\ref{fig:nonsym-last}.

\fig{images/nonsym30.png}{fig:nonsym-first}{2D nonsymmetric potential II:
eigenvalue 0}

\fig{images/nonsym303.png}{}{2D nonsymmetric potential II: eigenvalue 0, surface
plot}

\fig{images/nonsym31.png}{}{2D nonsymmetric potential II: eigenvalue 1}

\fig{images/nonsym313.png}{}{2D nonsymmetric potential II: eigenvalue 1, surface
plot}

\fig{images/nonsym32.png}{}{2D nonsymmetric potential II: eigenvalue 2}

\fig{images/nonsym323.png}{}{2D nonsymmetric potential II: eigenvalue 2, surface
plot}

\fig{images/nonsym33.png}{}{2D nonsymmetric potential II: eigenvalue 3}

\fig{images/nonsym333.png}{}{2D nonsymmetric potential II: eigenvalue 3, surface
plot}

\fig{images/nonsym34.png}{}{2D nonsymmetric potential II: eigenvalue 4}

\fig{images/nonsym343.png}{}{2D nonsymmetric potential II: eigenvalue 4, surface
plot}

\fig{images/nonsym35.png}{}{2D nonsymmetric potential II: eigenvalue 5}
% this is needed, otherwise latex cries on the fig below.
\clearpage

\fig{images/nonsym353.png}{}{2D nonsymmetric potential II: eigenvalue 5, surface
plot}

\fig{images/nonsym36.png}{}{2D nonsymmetric potential II: eigenvalue 6}

\fig{images/nonsym363.png}{}{2D nonsymmetric potential II: eigenvalue 6, surface
plot}

\fig{images/nonsym37.png}{}{2D nonsymmetric potential II: eigenvalue 7}

\fig{images/nonsym373.png}{}{2D nonsymmetric potential II: eigenvalue 7, surface
plot}

% this is needed, otherwise latex cries on the fig below.
\clearpage

\fig{images/nonsym38.png}{}{2D nonsymmetric potential II: eigenvalue 8}

\fig{images/nonsym383.png}{}{2D nonsymmetric potential II: eigenvalue 8, surface
plot}

\fig{images/nonsym39.png}{}{2D nonsymmetric potential II: eigenvalue 9}

\fig{images/nonsym393.png}{fig:nonsym-last}{2D nonsymmetric potential II: eigenvalue 9, surface plot}

\subsection{2D: nonsymmetric potential III}

Finally we calculate the Schr\"odinger equation for 20 atom nuclei in a circle.
See the figs \ref{fig:nonsym-first} --
\ref{fig:nonsym-last}.

\fig{images/nonsym40.png}{fig:nonsym-first}{2D nonsymmetric potential III:
eigenvalue 0}

\fig{images/nonsym40warp.png}{}{2D nonsymmetric potential III:
eigenvalue 0, surface plot}

\fig{images/nonsym44.png}{}{2D nonsymmetric potential III:
eigenvalue 4}

\fig{images/nonsym44warp.png}{}{2D nonsymmetric potential III:
eigenvalue 4, surface plot}

\fig{images/nonsym45.png}{}{2D nonsymmetric potential III:
eigenvalue 5}

\fig{images/nonsym45warp.png}{fig:nonsym2-first}{2D nonsymmetric potential III:
eigenvalue 5, surface plot}

\section{Density Functional Theory, Spherically Symmetric Solution}

\subsection{Pb: LDA}

\begin{lstlisting}
0: |F(x)|=32381.03274296
1: |F(x)|=5347.06647460
2: |F(x)|=2283.82989218
3: |F(x)|=148.26174998
4: |F(x)|=120.78098800
5: |F(x)|=84.68248231
6: |F(x)|=11.30008638
7: |F(x)|=3.65004163
8: |F(x)|=3.12545615
9: |F(x)|=1.44414848
10: |F(x)|=0.32879840
11: |F(x)|=0.10891716
12: |F(x)|=0.03456829
13: |F(x)|=0.01240870
14: |F(x)|=0.00774382
15: |F(x)|=0.00302906
16: |F(x)|=0.00081825
17: |F(x)|=0.00026270
18: |F(x)|=0.00007814
19: |F(x)|=0.00003516
1s( 2) j=l+1/2: -2901.078061
2s( 2) j=l+1/2: -488.8433352
2p( 6) j=l+1/2: -470.8777849
3s( 2) j=l+1/2: -116.526852
3p( 6) j=l+1/2: -107.950391
3d(10) j=l+1/2: -91.88992429
4s( 2) j=l+1/2: -25.75333021
4p( 6) j=l+1/2: -21.99056413
4d(10) j=l+1/2: -15.03002657
4f(14) j=l+1/2: -5.592531664
5s( 2) j=l+1/2: -4.206797624
5p( 6) j=l+1/2: -2.941656967
5d(10) j=l+1/2: -0.9023926829
6s( 2) j=l+1/2: -0.3571868295
6p( 2) j=l+1/2: -0.1418313263
\end{lstlisting}

This is agrees with the NIST reference calculation to all decimal digits.

\subsection{Pb: RLDA}

\begin{lstlisting}
0: |F(x)|=41056.67822797
1: |F(x)|=7926.04357205
2: |F(x)|=2858.46358638
3: |F(x)|=598.63802038
4: |F(x)|=268.50208068
5: |F(x)|=30.05744371
6: |F(x)|=27.56292000
7: |F(x)|=11.22084649
8: |F(x)|=4.78645898
9: |F(x)|=0.53300950
10: |F(x)|=0.13956963
11: |F(x)|=0.07821891
12: |F(x)|=0.06505839
13: |F(x)|=0.02021479
14: |F(x)|=0.00240256
15: |F(x)|=0.00132181
16: |F(x)|=0.00079552
17: |F(x)|=0.00018579
18: |F(x)|=0.00000838
19: |F(x)|=0.00000584
1s( 2) j=l+1/2: -3209.51946
2s( 2) j=l+1/2: -574.1825655
2p( 6) j=l-1/2: -551.7234408
2p( 6) j=l+1/2: -472.3716103
3s( 2) j=l+1/2: -137.8642241
3p( 6) j=l-1/2: -127.6789451
3p( 6) j=l+1/2: -109.9540395
3d(10) j=l-1/2: -93.15817605
3d(10) j=l+1/2: -89.36399096
4s( 2) j=l+1/2: -31.15015728
4p( 6) j=l-1/2: -26.73281564
4p( 6) j=l+1/2: -22.38230707
4d(10) j=l-1/2: -15.1647618
4d(10) j=l+1/2: -14.3484973
5s( 2) j=l+1/2: -5.225938506
4f(14) j=l-1/2: -4.960490099
4f(14) j=l+1/2: -4.775660273
5p( 6) j=l-1/2: -3.710458943
5p( 6) j=l+1/2: -2.889127431
5d(10) j=l-1/2: -0.8020049565
5d(10) j=l+1/2: -0.7070299184
6s( 2) j=l+1/2: -0.4209603386
6p( 2) j=l-1/2: -0.1549640727
\end{lstlisting}

This is agrees within $12\%$ to the NIST reference calculation, the difference
being caused probably by a different exchange and correlation potential
approximation in our code in NIST.

\subsection{B: LDA}

\begin{lstlisting}
1:  |F(x)|=467.33470427
2:  |F(x)|=39.46088238
3:  |F(x)|=5.59717305
4:  |F(x)|=3.09300726
5:  |F(x)|=2.04909614
6:  |F(x)|=0.09754169
7:  |F(x)|=0.06773803
8:  |F(x)|=0.04587578
9:  |F(x)|=0.00592044
10:  |F(x)|=0.00382678
11:  |F(x)|=0.00232014
12:  |F(x)|=0.00005561
13:  |F(x)|=0.00002714
14:  |F(x)|=0.00001809
15:  |F(x)|=0.00000042
16:  |F(x)|=0.00000023
17:  |F(x)|=0.00000014
18:  |F(x)|=0.00000001
19:  |F(x)|=0.00000000
20:  |F(x)|=0.00000000
1s( 2) j=l+1/2: -6.564347081
2s( 2) j=l+1/2: -0.3447010093
2p( 1) j=l+1/2: -0.1366031499
\end{lstlisting}

Agrees with NIST to all decimal places.

\subsection{B: RLDA}

\begin{lstlisting}
0: |F(x)|=485.06815695
1: |F(x)|=103.13739852
2: |F(x)|=34.94893789
3: |F(x)|=18.15071235
4: |F(x)|=1.19766447
5: |F(x)|=0.10974802
6: |F(x)|=0.07628667
7: |F(x)|=0.02450426
8: |F(x)|=0.00430650
9: |F(x)|=0.00193614
10: |F(x)|=0.00042938
11: |F(x)|=0.00010306
12: |F(x)|=0.00002872
13: |F(x)|=0.00001073
14: |F(x)|=0.00000217
15: |F(x)|=0.00000090
16: |F(x)|=0.00000028
17: |F(x)|=0.00000015
18: |F(x)|=0.00000002
19: |F(x)|=0.00000001
1s( 2) j=l+1/2: -6.56282977
2s( 2) j=l+1/2: -0.3447247582
2p( 1) j=l-1/2: -0.1366103284
\end{lstlisting}

Agrees with NIST to 3 decimal places after the decimal dot, the difference also
probably causes by a different exchange and correlation potential
approximation.
